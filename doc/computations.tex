\chapter{Computations (Extended Attribute Evaluation)}\indexmain{computations}\indexmain{attribute evaluation}
\label{cha:computations}

In this chapter we'll have a look at the computations available in the rule language.
Their most important construct is the attribute assignment to assign new values to graph element attributes.
Besides this, further types of assignments are available, and several control flow constructs as known from imperative programming languages of the C-family.
Furthermore, container methods may be called (more on this in the following chapter \ref{cha:container}), 
and global graph functions may be called (more on this in the following chapter \ref{cha:graph}).

\begin{rail}
  Statement:
      Assignment
    | CompoundAssignment
    | IndexedAssignment
    | ContainerMethodCall
    | GlobalGraphFunctionCall
    | VisitedAssignment
    | CallOrMiscGlobalFunctionCall
    | EmbeddedExec
    | LocalVariableDecl
    | Decision
    | WhileLoop
    | ForLoop
    | EarlyExit
    | ReturnStatement
    ;
\end{rail}\ixnterm{Statement}\label{computationstatemet}

%explain method calls, function calls, compound assignment, for loops in the container and graph chapters.
%The \emph{Assignment} will be explained below, the \emph{CompoundAssignment}, \emph{VisitedAssignment} and \emph{ContainerMethodCall} clauses will be introduced in chapter \ref{cha:container} and \ref{cha:graph}.


%%%%%%%%%%%%%%%%%%%%%%%%%%%%%%%%%%%%%%%%%%%%%%%%%%%%%%%%%%%%%%%%%%%%%%%%%%%%%%%%%%%%%%%%%%%%%%%%
\section{Assignments} \label{sub:assignments}\indexmain{assignments}

\begin{rail}
  Assignment:
	  NodeOrEdge '.' Ident '=' Expression ';' |
	  '::' GlobalVarIdent '.' Ident '=' Expression ';' |
	  Variable '=' Expression ';' |
	  '::' GlobalVarIdent '=' Expression ';'
	;
\end{rail}\indexmain{attribute evaluation}\ixnterm{Assignment}

Evaluation statements have \indexedsee{imperative}{attribute evaluation} semantics.
In particular, \GrG\ does not care about data dependencies (you must care for them!).
Assignment is carried out using value semantics, even for entities of container or \texttt{string} type.
The only exception is the type \texttt{object}, there reference semantics is used.
You can find out about the available \emph{Expression}s in chapter \ref{cha:typeexpr}.

%compound assignments are given in following chapters

%%%%%%%%%%%%%%%%%%%%%%%%%%%%%%%%%%%%%%%%%%%%%%%%%%%%%%%%%%%%%%%%%%%%%%%%%%%%%%%%%%%%%%%%%%%%%%%%
\section{Local Variable Declarations} 

\begin{rail} 
  LocalVariableDecl: 
	'def' Name ':' Type ('=' Expression)? ';' |
	'def' '-' Name ':' Type '->' ('=' Expression)? ';' |
	'def' 'var' Name ':' Type ('=' Expression)? ';'
	;
\end{rail}\ixnterm{LocalVariableDecl}

Local variables can be defined with the syntax known for local def pattern variables, as already introduced in \ref{sec:localvarorderedevalyield}, i.e. \texttt{def var name:type} for variables, or \texttt{def name:type} for nodes, or \texttt{def -name:type->} for edges.
At their definition, an initializing expression may be given.

%%%%%%%%%%%%%%%%%%%%%%%%%%%%%%%%%%%%%%%%%%%%%%%%%%%%%%%%%%%%%%%%%%%%%%%%%%%%%%%%%%%%%%%%%%%%%%%%
\section{Calls and Misc. Global Functions} 

You may call any of the functions from the expressions that were already introduced or that will be introduced just for its side effects, throwing away the result.
Besides there are functions to emit text or record graph changes available in the computations: 

\begin{description}
\item[\texttt{emit(.)}] writes the argument value, typically a test string, to stdout, or to a file if output was redirected. 
\item[\texttt{record(.)}] writes the argument value, typically a text string, to the graph change record.
\end{description}

%%%%%%%%%%%%%%%%%%%%%%%%%%%%%%%%%%%%%%%%%%%%%%%%%%%%%%%%%%%%%%%%%%%%%%%%%%%%%%%%%%%%%%%%%%%%%%%%
\section{Embedded Exec} 

As in the rules (cf. chapter \ref{cha:imperativeandstate}), \texttt{exec} statements may be embedded.
They are executing the graph rewrite sequence (see chapter \ref{cha:xgrs}) contained, with access to the variables defined outside.
This means for one reading this variables, but for the other assigning to that variables with \texttt{yield}.
The embedded \texttt{exec}s allow to apply rules from the computations, while the sequence computations may call defined computations, mixing and merging declarative rule-based graph rewriting (pattern matching and dependent modifications) with traditional graph programming; in addition to the computations as such, which are already going into that direction, enriching rules with general graph programming.

\begin{description}
\item[\texttt{exec(.)}] executes the graph rewrite sequence given. 
\end{description}

%%%%%%%%%%%%%%%%%%%%%%%%%%%%%%%%%%%%%%%%%%%%%%%%%%%%%%%%%%%%%%%%%%%%%%%%%%%%%%%%%%%%%%%%%%%%%%%%
\section{Control flow} \label{sub:controlflow}\indexmain{controlflow}

Based on the value of a boolean expression different computations can be executed conditionally with the \texttt{if} and \texttt{else} statements. Please not that the braces are mandatory.

\begin{rail} 
  Decision: IfPart ((ElseIfPart)*()) (ElsePart)? ;
	IfPart: 'if' '(' BoolExpr ')' lbrace (Statement*) rbrace ;
	ElseIfPart: 'else' 'if' '(' BoolExpr ')' lbrace (Statement*) rbrace ;
	ElsePart: 'else' lbrace (Statement*) rbrace ;
\end{rail}\ixnterm{Decision}\ixkeyw{if}\ixkeyw{else}

Computations may be executed repeatedly by the \texttt{while} and \texttt{do-while} statements. Please note that the braces are mandatory and that the \texttt{do-while} loop does not need a terminating semicolon.

\begin{rail} 
  WhileLoop: 
	'while' '(' BoolExpr ')' lbrace (Statement*) rbrace |
	'do' lbrace (Statement*) rbrace 'while' '(' BoolExpr ')' 
	;
\end{rail}\ixnterm{Loop}\ixkeyw{while}\ixkeyw{do}

%\section{Container iteration}

The containers introduced in chapter \ref{cha:container} may be iterated over with a for loop, one assigning the current value to a variable, the other assigning the current index and the current value to a variable.
\begin{rail}
  ForLoop: (ForHeader | IndexedForHeader) lbrace (Statement*) rbrace;
  ForHeader: 'for' '(' Var ':' Type 'in' ContainerVar ')';
  IndexedForHeader: 'for' '(' IndexVar ':' Type '->' Var ':' Type 'in' ContainerVar ')' ;
\end{rail}\ixkeyw{in}\ixkeyw{for}\ixnterm{ForLoop}

Loop body execution may be exited early or continued with the next iteration with the \texttt{break} and \texttt{continue} statements. Such a statement must not occur outside of a loop.

\begin{rail} 
  EarlyExit: 
	'break' ';' |	'continue' ';'
	;
\end{rail}\ixnterm{EarlyExit}\ixkeyw{break}\ixkeyw{continue}

%%%%%%%%%%%%%%%%%%%%%%%%%%%%%%%%%%%%%%%%%%%%%%%%%%%%%%%%%%%%%%%%%%%%%%%%%%%%%%%%%%%%%%%%%%%%%%%%
\section{Computation definition} \label{sub:compdef}\indexmain{computation definition}

Computations that are occuring frequently may be factored out into a computation definition given in the header of a rule file.
\begin{rail} 
  ComputationDefinition: 
	Name '(' (Parameter * ',') ')' ':' ReturnType \\
	lbrace (Statement+) rbrace
  ;
  Parameter: IdentDecl ':' NodeType |
  '-' IdentDecl ':' EdgeType '->' |
  ('var' | 'ref') IdentDecl ':' VarType
	;
\end{rail}\ixnterm{Parameter}\ixkeyw{var}\ixkeyw{ref}

The computation definition must return a value with a return statement (so there must be at least one return statement).
The type of the one return value must be compatible to the return type specified in the computation definition.
A return statement must not occur outside of a computation definition.

\begin{rail}
  ReturnStatement: 'return' '(' Expression ')' ';' ;
\end{rail}\ixkeyw{return}\ixnterm{ReturnStatement}

A such defined computation may then be called as an expression atom from anywhere in the rule language file where an expression is required; or even from the sequence computations.

\begin{rail}
  CallExpr: Name '(' (Expression * ',') ')' ;
\end{rail}\ixnterm{CallExpr}
