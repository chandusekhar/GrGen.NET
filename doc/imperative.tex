\chapter{Embedded Sequences and Storages}\indexmain{imperativeandstate}
\label{cha:imperativeandstate}

In this chapter we'll have a look at language constructs which allow to emit text from rules/subpatterns and which allow to execute a graph rewrite sequence at the end of a rule invocation.
The ability to execute a sequence at the end of a rule invocation allows to combine rules and build complex rules.
Furthermore we'll have a look at language constructs which allow to build transformations by introducing state as a communications means between several rule applications.
Finally we apply these techniques to accomplish certain common transformation tasks.

%%%%%%%%%%%%%%%%%%%%%%%%%%%%%%%%%%%%%%%%%%%%%%%%%%%%%%%%%%%%%%%%%%%%%%%%%%%%%%%%%%%%%%%%%%%%%%%%
\section{Imperative Statements}\indexmain{imperative statements} 
\label{sct:imperative}

%-----------------------------------------------------------------------------------------------
\subsection{Exec and emit in rules}

\begin{rail}  
  ExecStatement: 'emit' '(' StringExpr ')' ';' |
    'exec' '(' RewriteSequence ')' \\ ('->' ((IdentDecl ':' NodeType | '-' IdentDecl ':' EdgeType '->') + ','))? ';'
	;
\end{rail}\ixkeyw{emit}\ixkeyw{exec}\ixnterm{ExecStatement}
The statements \texttt{emit} and \texttt{exec} enhance the declarative rewrite part by imperative clauses.
That means, these statements are executed in the same order as they appear within the rule.
The execution statements take place after all the rewrite operations are done, i.e.\ they operate on the modified host graph.
However, \emph{attribute} values of deleted graph elements are still available for reading.
The \texttt{eval} statements are executed before the execution statements, i.e.\ the execution statements work on the recalculated attributes.
\begin{description}
  \item[XGRS Execution] The \texttt{exec} statement executes a graph rewrite sequence, which is a composition of graph rewrite rules. Yielded graph elements may be included into the RHS pattern, but as of now they are only accessible from the \texttt{return} statement. See Chapter~\ref{cha:xgrs} for a description of graph rewrite sequences. The \texttt{exec} statement is one of the means available in \GrG~to build complex rules and split work into several parts, see \ref{sub:copymergesplit} for a discussion of this topic.
  \item[Text Output] The \texttt{emit} statement prints a string to the currently associated output stream (default is \texttt{stdout}). See Chapter~\ref{cha:typeexpr} for a description of string expressions.
  For emitting in between the emits from subpatterns, there is an additional \texttt{emithere} statement available.
\end{description}

\begin{figure}[htbp]
\begin{example}
	The following example works on a hypothetical network flow.
	We don't define all the rules nor the graph meta model.
	It's just about the look and feel of the \texttt{exec} and \texttt{emit} statements
	\begin{grgen}
rule AddRedundancy
{
  s: SourceNode;
  t: DestinationNode;	
  modify {
    emit ("Source node is " + s.name + ". Destination node is " + t.name + ".");
    exec ( (x:SourceNode) = DuplicateNode(s) & ConnectNeighbors(s, x)* );
    exec ( [DuplicateCriticalEdge] );
    exec ( MaxCapacityIncidentEdge(t)* );
    emit ("Redundancy added");
  }
}  
	\end{grgen}
\end{example}
\end{figure}

\begin{example}
This is an example for returning elements yielded from an \texttt{exec} statement.
The results of the rule \texttt{bar} are written to the variables \texttt{a} and \texttt{b};
The \texttt{yield} statement is similar to a return but does not jump out. 
The anonymous output variables written in the yield statement are assigned in the given order to the pattern variables specified after the assign to operator \verb#->#.

	\begin{grgen}
rule foo : (A,B)
{
  modify {
    exec( (a,b)=bar && yield(a,b) ) -> u:A, v:B;
    return(u,v)
  }
}  
	\end{grgen}
\end{example}

%-----------------------------------------------------------------------------------------------
\subsection{Deferred exec and emithere in nested and subpatterns}

\begin{rail}  
  SubpatternExecEmit: 
		'emithere' '(' StringExpr ')' ';' |
		'emit' '(' StringExpr ')' ';' |
		'exec' '(' RewriteSequence ')' ';'
	;
\end{rail}\ixnterm{SubpatternExecEmit}\ixkeyw{emithere}

The statements \texttt{emit}, \texttt{emithere} and \texttt{exec} enhance the declarative nested rewrite part by imperative clauses.
That means, these statements are executed in the same order as they appear within the rule.
The \texttt{emithere}-statements are executed before the emit and exec statements,
and in the order in between the subpattern rewrite applications they are specified syntactically.
The \texttt{exec}-statements listed here are a slightly different version of the \texttt{exec}-statements from the \emph{ExecStatement} introduced in \ref{replclause}.
They are only available in the rewrite parts of subpatterns or nested alternatives/iterateds,
but not in the rewrite part of rules as the original embedded sequences.
They are executed after the original rule calling them was executed,
so they can't get extended by \texttt{yield}s, 
as the containing rule is not available any more when they get executed.

\begin{note}
The embedded sequences are executed after the top-level rule which contains them in a nested pattern or in a used  subpattern was executed; they are \emph{not} executed during subpattern rewriting. They allow you to put work you can't do while executing the rule proper (e.g. because an element was already matched and is now locked due to the isomorphy constraint) to a waiting queue which gets processed afterwards --- with access to the elements of the rule and contained parts which are available when the rule gets executed. Or to just split the work into several parts, reusing already available functionality, see \ref{sub:copymergesplit} for a discussion on this topic.
\end{note}

\begin{example}
	The exec from \texttt{Subpattern sub} gets executed after the exec from \texttt{rule caller} was executed.
	\begin{grgen}
rule caller
{
  n:Node;
  sub:Subpattern();
    
  modify {
    sub();
    exec(r(n));
  }
}  
pattern Subpattern
{
  n:Node;
  modify {
    exec(s(n));
  }
}
	\end{grgen}
\end{example}

\begin{example}
	This is an example for emithere, showing how to linearize an expression tree in infix order.
	\begin{grgen}
pattern BinaryExpression(root:Expr)
{
  root --> l:Expr; le:Expression(l);
  root --> r:Expr; re:Expression(r);
  root <-- binOp:Operator;
  
  modify {
    le(); // rewrites and emits the left expression
    emithere(binOp.name); // emits the operator symbol in between the left tree and the right tree
    re(); // rewrites and emits the right expression
  }
}  
	\end{grgen}
\end{example}


%%%%%%%%%%%%%%%%%%%%%%%%%%%%%%%%%%%%%%%%%%%%%%%%%%%%%%%%%%%%%%%%%%%%%%%%%%%%%%%%%%%%%%%%%%%%%%%% 
\section{Reading, writing, and managing state}\indexmain{visited flag}\indexmain{storage}

Besides the normal scalar variables from the rule application control language there are sets and map valued storages available for communicating some information between rule applications;
additionally there are several visited flags per graph element available for this means.
Both allow to split a transformation into passes mediated by their state.

\begin{note}
The storage sets are more efficient in case the count of elements of interest is a good deal smaller than the number elements in the graph; they are looked up in the set, in contrast to the visited flags, which are used by enumerating all available graph elements, filtering according to visited state.
The visited flags on the other hand are much more memory efficient (unless you use more than a few of them at the same time).
\end{note}

%-----------------------------------------------------------------------------------------------
\subsection{Storage and visited flag handling in the sequences}\label{sec:storages}

Storages are variables of set or map type (cf. \ref{sec:builtintypes}) used in the sequences or rules handed in via parameters;
in contrast to the sets or maps in the graph model, their elements can be of node or edge type. 
Storing nodes and edges is in fact their primary usage. 
They allow to decouple processing phases: the first run collects all graph elements relevant for the second run which consists of a sequence executed for each graph element in the set.
A further difference to the sets and maps in the rewrite rules is that they only offer imperative addition and removal instead of union, intersection, difference and construction.
The splitting of transformations into passes mediated by set/map valued global variables allows for subgraph copying without model pollution, cf. \ref{subsub:copystructure}; please have a look at \ref{sub:copymergesplit} regarding a discussion on when to use which transformation combinators.
 
\begin{rail}
  VariableHandling: 
    Variable (SetMapTypeDecl)? '=' RHS |
    SetVariable '.' ( 'add' '(' Value ')' | 'rem' '(' Value ')' | 'clear' '(' ')' ) |
    MapVariable '.' ( 'add' '(' Key ',' Value ')' | 'rem' '(' Key ')' | 'clear' '(' ')' ) |
	Variable '.' Attribute '=' Variable |
	Varible 'in' SetMapVariable |
	Variable '=' MapVariable '[' Variable ']'
    ;
  SetMapTypeDecl: 
    ':' ('set' '<' Type '>' | 'map' '<' KeyType ',' ValueType '>')
    ;
\end{rail}\ixnterm{VariableHandling}\ixnterm{SetMapTypeDecl}\ixkeyw{in}\indexmain{map}\indexmain{set}\makeatother

\begin{rail}
  RHS:
    SetMapCreation |
	Variable |
	Variable '.' Attribute |
    SetVariable '.' ( size '(' ')' | empty '(' ')' ) |
    MapVariable '.' ( size '(' ')' | empty '(' ')' )
    ;
  SetMapCreation:
	'set' '<' Type '>' lbrace rbrace |
    'map' '<' KeyType ',' ValueType '>' lbrace rbrace
	;
\end{rail}\ixnterm{RHS}\ixnterm{SetMapCreation}\makeatother

A set/map must be created and assigned to a variable before it can be used.

\begin{example}
\begin{grgen}
x=set<NodeTypeA>{}
y:map<Node,Edge> = map<Node,Edge>{}
\end{grgen}
The first line declares or references a variable \texttt{x} (without static type) and assigns the newly created, empty set of type \texttt{set<NodeTypeA>} to it as value.
The second line declares a variable \texttt{y} of type \texttt{map<Node,Edge>} and assigns the newly created, empty map of the same type to it as value.
\end{example}

\noindent There are several operations on set variables available in method call notation, these are:

\begin{description}
\item[Set addition:] \texttt{s.add(v)} adds the value \texttt{v} to the set \texttt{s}, succeeds always.
\item[Set removal:] \texttt{s.rem(v)} removes the value \texttt{v} from the set \texttt{s}, succeeds always.
\item[Set clearing:] \texttt{s.clear()} removes all values from the set \texttt{s}, succeeds always.
\end{description}

\noindent Very similar operations are available on map variables:

\begin{description}
\item[Map addition:] \texttt{m.add(k,v)} adds the pair (\texttt{k},\texttt{v}) to the map \texttt{m}, succeeds always.
\item[Map removal:] \texttt{m.rem(k)} removes the pair (\texttt{k},unknown) from the map \texttt{m}, succeeds always.
\item[Map clearing:] \texttt{m.clear()} removes all key-value-pairs from the map \texttt{m}, succeeds always.
\end{description}

\noindent There are further operations which are only available in variable assignments:

\begin{description}
\item[Size assignment:] \texttt{v=sm.size()} writes the number of entries in the set or map \texttt{sm} to the variable \texttt{v}, succeeds always.
\item[Emptyness assignment:] \texttt{v=sm.empty()} writes to the variable \texttt{v} whether the set or map \texttt{sm} is empty,\\ succeeds always.
\item[Map lookup assignemt:] \texttt{v=m[el]} assigns the result of the map lookup to the variable \texttt{v},\\succeeds iff \texttt{el} is contained in \texttt{m},\\ fails otherwise, not touching the variable \texttt{v}.
\end{description}

\begin{rail}
  RewriteFactor:
    Var 'in' SetVar |
    Var 'in' MapVar |
    'for' lbrace Var 'in' SetVar ';' RewriteSequence rbrace |
    'for' lbrace Var '->' Var 'in' MapVar ';' RewriteSequence rbrace
\end{rail}\ixkeyw{in}\ixkeyw{for}\ixnterm{RewriteFactor}

\noindent Handling of the storages is completed by the rewrite factors for membership query and storage iteration.
The binary operator \texttt{el in sm} checks for set/map membership; it returns true if \texttt{el} is contained in the set or the domain of the map, otherwise false.
The \texttt{for} command iterates over all elements in the set or all key-value pairs in the map and executes for each element / key-value pair the nested graph rewrite sequence; it completes successfully iff all sequences were executed successfully (an empty set/map causes immediate successful completion).

\begin{example}
The following XGRS is a typical storage usage.
First an empty set \texttt{x} is created, which gets populated by an rule \texttt{t} executed iteratedly, returning a node which is written to the set.
Then another rule is executed iteratedly for every member of the set doing the main work, and finally the set gets cleared to prevent memory leaks or later mistakes.
If the graph should stay untouched during set filling you may need \texttt{visited} flags to prevent endless looping.
\verb#x=set<Node>{} ;> ( (v)=t() && x.add(v) )+ && for{v in x; r(v)} <; x.clear()#
Handing in the storage to the rule, and using the set \texttt{add} method as introduced down below in \ref{sct:imperative} within the rule to fill the storage, allows to shorten the sequence to:\\
\verb#x=set<Node>{} ;> ( t(x) )+ && for{v in x; r(v)} <; x.clear()#
\end{example}

\begin{note}
The set/map over which the for loop iterates must stay untouched during iteration.
\end{note}

%+++++++++++++++++++++++++++++++++++++++++++++++++++++++++++++++++++++++++++++++++++++++++++++++
\vspace{5mm}
\hrule
\vspace{5mm}

\begin{rail}
  VisitedFlagsManagement:
    Variable (':' Type)? '=' \\('valloc' '(' ')' | GraphElement '.' 'visited' '[' Variable ']') |
    'vfree' '(' Variable ')' |
    GraphElement '.' 'visited' '[' Variable ']' '=' (Variable|BoolLit) |
    GraphElement '.' 'visited' '[' Variable ']' |
    'vreset' '(' Variable ')'
\end{rail}\ixnterm{VisitedFlagsManagement}\ixkeyw{valloc}\ixkeyw{vfree}\ixkeyw{visited}\ixkeyw{vreset}
The visited flags are stored in some excess bits of the graph elements, if this pool is exceeded they are stored in additional dictionaries, one per visited flag requested.
Due to this flags must get allocated/deallocated, and all flag related operations require an integer number -- the flag id -- specifying the flag to operate on (with exception of the allocation operation returning this flag id).
The operations always return true as sequence results (with exception of the operation reading the flag, it fails iff the visited flag is not set for the graph element);
if you try to access a not previously allocated visited flag, an exception is thrown.
The operations managing the visited flags are:
\begin{description}
\item[Flag allocation:] By \texttt{valloc}\label{allocvisitflag} -- allocates space for a visited flag in the elements of the graph and returns the id of the visited flag (integer number), starting at 0.
Afterwards, the visited flag of the id can be read and written within the rules by the \texttt{visited}-expression and the \texttt{visited}-assignment,
as well as by the \texttt{visited} flag reading and writing rewrite factors.
The first visited flags are stored in some excess bits of the graph elements and are thus essentially for free,
but if this implementation defined space is used up completely, the information is stored in graph element external dictionaries.
\item[Flag deallocation:] By \texttt{vfree} -- frees the space previously allocated for the visited flag; afterwards you must not access it anymore. 
The value stored in the variable must be of integer type, stemming from a previous allocation.
\item[Flag writing:] By \texttt{e.visited[f] = b} -- sets the visited status of the flag given by the flag id variable \texttt{f} of the graph element \texttt{e} to the given boolean value \texttt{b}; visited flags are normally written by rules of the rule language.
\item[Flag reading:] By \texttt{e.visited[f]} -- returns the visited status of the flag given by the flag id variable \texttt{f} of the graph element \texttt{e}; visited flags are normally read by tests and rules of the rule language.
\item[Flag resetting:] By \texttt{vreset} -- resets the visitor flag given by the flag id variable in all graph elements.
\end{description}


%-----------------------------------------------------------------------------------------------
\subsection{Storage and visited flag access in the rules} \label{sub:storageaccess}\indexmain{storage access}

Storages can be used in the rule control language as introduced above \ref{sec:storages}, they can get filled or emptied in the rules as defined here \ref{replstmt}, a discussion about their usage is given here \ref{sub:copymergesplit}.
In the pattern part you may ask for an element to get bound to an element from a storage; this is syntactically specified by giving the storage enclosed in left and right angles.
You may ask for an element to get bound to the value element queried from a storagemap by a key graph element; this is syntactically specified by giving the storagemap indexed by the key graph element enclosed in left and right angles.
If the type of the element retrieved from the storage is not compatible to the type of the pattern element specified,
or if the storage is empty, or if the key element is not contained in the storagemap, matching fails.

\begin{rail}
  StorageAccess: '<' Ident '>' | '<' Ident '[' Ident ']' '>';
\end{rail}\ixnterm{StorageAccess}

\begin{example}
Queries the graph for the neighbouring cities to the cities contained in the storageset.
\begin{grgen}
test neighbour(ref startCities:set<City>) : City
{
    :City<startCities> -:Street-> n:City;
    return(n);
}
\end{grgen}
\end{example}

\begin{example}
Queries for the neighbour of the neighbour of a city matched. With the first neigbouring relation queried from the storagemap assumed to contain the neighbouring relation of some cities of interest, and the second neighbouring relation queried from the graph.
\begin{grgen}
test neighbourneighbour(ref neighbours:map<City, City>) : City
{
    someCity:City;
    nc:City<neighbours[someCity]> -:Street-> nnc:City;
    return(nnc);
}
\end{grgen}
\end{example}

These were the storage queries available in the pattern part; 
additionally you can query the storage sets and maps in the \texttt{if} attribute evaluation clause,
with the set and map expressions as given in the Types and Expressions chapter \ref{cha:typeexpr}.
Furthermore you can add and remove elements from the storages in the \texttt{eval} clause of the rewrite part.

\begin{rail}    
  SetMapStateChange:
   	SetParameter '.' ( ( 'add' '(' Expr ')') | ( 'rem' '(' Expr ')' ) ) |
	MapParameter '.' ( ( 'add' '(' KeyExpr ',' ValueExpr ')' ) | ( 'rem' '(' KeyExpr ')' ) )
	;
\end{rail}\ixnterm{SetMapStateChange}

The by-ref parameters can be operated upon by the set/map state change methods,
which allow to only partially change the set/map by adding and removing elements resp. pairs of elements;
they are especially useful for sets or maps containing nodes or edges.

\begin{example}
The set/map state change methods \texttt{add} and \texttt{rem} allow to add graph elements to storages or remove graph elements from storages, i.e. sets or maps holding nodes and edges used for rewrite in the calling sequence (cf. \ref{sec:storages}).
This way you can write transformations consisting of several passes with one pass operating on nodes/edges determined in a previous pass,
without the need to mark the element in the graph by helper edges or visited flags.
	\begin{grgen}
rule foo(ref storage:set<Node>)
{
  n:Node;
  modify {
    eval {
      storage.add(n);
    }
  }
}  
	\end{grgen}
\end{example}

%+++++++++++++++++++++++++++++++++++++++++++++++++++++++++++++++++++++++++++++++++++++++++++++++
\vspace{5mm}
\hrule
\vspace{5mm}

The visited flag queries available in the \texttt{if} attribute evaluation clause of the pattern part are given in the Types and Expressions chapter \ref{cha:typeexpr}.
Additionally you can set them in the \texttt{eval} clause of the rewrite part.

\begin{rail}    
  VisitedAssignment:
	NodeOrEdge '.' 'visited' ('[' FlagNumber ']')? '=' BoolExpr
	;
\end{rail}\ixnterm{VisitedAssignment}\ixkeyw{visited}

The \texttt{visited} flag assignment sets the \texttt{boolean} status of the \indexed{visited flag} of the given number for the given graph element.
If no flag number is given, the default number for the first visited flag of 0 is used.
Make sure to allocate \ref{allocvisitflag}/\ref{apiallocvisitflag} visited flags before you try to use them 
(and deallocate them afterwards, as they are a sparse resource stored in some excess bits of the graph elements, or in some dictionary if the needed number of flags exceeds the number of available bits per graph element.)


%%%%%%%%%%%%%%%%%%%%%%%%%%%%%%%%%%%%%%%%%%%%%%%%%%%%%%%%%%%%%%%%%%%%%%%%%%%%%%%%%%%%%%%%%%%%%%%%
\section{Merge and Split, Node Replacement Grammars, and Subgraph copying}\label{sub:copymergesplit}

In this section we'll have a look at several common graph rewriting tasks and how to accomplish them in \GrG.
They could be offered directly by some dedicated operators,
but these would need so much customization to be useful in the different situations one needs them,
that we decided against dedicated operators; 
instead it is on you to program the version you need yourself by combining language constructs and rules.

There are two-and-a-half means available in the \GrG-rule language to build combined, complex rules:
\begin{description}
	\item[nested and subpatterns]
which allow to match and rewrite complex patterns built in a structured way piece by piece.
With different pieces connected together, pieces to decide in between, and pieces which appear repeatedly. 
	\item[embedded graph rewrite sequences]
for deferred rule execution to do work later on you can't do while executing the rule proper (e.g. because an element was already matched and is now locked due to the isomorphy constraint); or for splitting work into several parts, reusing already available functionality.
	\item[storages and storagemaps]
to store elements collected in one run and reuse them as input in another run, i.e. using multi-value variables to break up the transformation into passes. (This is the half point, as they are not a means to build a complex rule, but a means to communicate between rules building complex transformations.)
\end{description}

\noindent In the following we'll employ them to merge and split nodes, emulate node replacement grammars, and copy substructures.

\begin{note}
You should use the means to build complex transformations in the order given, first try to solve the problem with nested and subpatterns, if they don't get the job done (often because of the isomorphy constraint locking of already matched elements) or feel awkward then embedded graph rewrite sequences, and if these are still inadequate then storages and storagemaps; i.e. from declarative-local to imperative-global. This helps in keeping the code readable and easily adaptable. And don't forget the last resort if you must solve a task so complex that rules, control and storages are not sufficient: adding helper nodes, edges, or attributes to the graph model and the graph itself (with the visited flags being a light version of this tactic). Note: storages and storagemaps might be useful when you are experiencing performance problems, replacing a search in the graph by a lookup in a dictionary. But beware of elements already deleted from the graph still hanging out in your storage because you forgot to remove them.
\end{note}

\subsubsection*{Merge and split nodes}\indexmain{merge node}\indexmain{split node}
Merging a node \texttt{m} into a node \texttt{n} means transferring all edges from node \texttt{m} to node \texttt{n}, then deleting node \texttt{m}. 
Splitting a node \texttt{m} off from a node \texttt{n} means creating a node \texttt{m} and transferring some edges from node \texttt{n} to \texttt{m}.

In both cases there are a lot of different ways how to handle the operation exactly:
Maybe only incoming or only outgoing edges, or only edges of a certain type \texttt{T} or only edges not of type \texttt{T}; maybe the node \texttt{n} is to be retyped, maybe the edges are to be retyped. 
But common is the transferring of edges; this can be handled succintly by an \texttt{iterated} statement and the \texttt{copy} operator.
In case the node opposite to an edge may be incident to several such edges, one must use an \texttt{exec} instead, as every iteration locks the matched entities, so they can't get matched twice. Not needing the opposite node one could simply leave it unmentioned in the pattern, only referencing node \texttt{n} or \texttt{m} and the edge, but unfortunately we need the opposite node so we can connect the edge copy to it.

Now we'll have a look at an example for node merging: T1-T2 analysis from compiler construction is used to find out whether a control flow graph of a subroutine is reducible, i.e. all loops are natural loops. All loops being natural loops is a very useful property for many analyses and optimizations. The analysis is split into two steps, T1 removes reflexive edges, T2 merges a control flow successor into its predecessor iff there is only one predecessor available. These two steps are iterated until the entire graph is collapsed into one node which means the control flow is reducible, or execution gets stuck before, in which case the control flow graph is irreducible.
The analysis is defined on simple graphs, i.e. if two control flow edges between two basic block nodes appear because of merging they are seen as one, i.e. they are automatically fused into one. As \GrG~is built on multigraphs we have to explicitely do the edge fusion in a further step T3.

First let us have a look at T1 and T3, which are rather boring ... ehm, straight forward:

  \begin{example}
    \begin{grgen}
rule T1
{
  n:BB -:cf-> n;
  
  replace {
    n; // delete relexive edges
  }
}

rule T3
{
  pred:BB -first:cf-> succ:BB;
  pred    -other:cf-> succ;
  
  modify { // kill multiedges
    delete(other);
  }
}
    \end{grgen}
  \end{example}

The interesting part is T2, this is the first version using an iterated statement:

  \begin{example}
    \begin{grgen}
rule T2
{
  pred:BB -e:cf-> succ:BB;
  negative {
    -e->;
    -:cf-> succ; // if succ has only one predecessor
  }
  iterated {
    succ -ee:cf-> n:BB;
    
    modify { // then merge succ into this predecessor
      pred -:copy<ee>-> n; // copying the succ edges to pred
    }
  }
  
  modify { // then merge succ into this predecessor
    delete(succ);
  }
}
    \end{grgen}
  \end{example}

In case a control flow graph would be a multi-graph, with several control flow edges between two nodes, one would have to use an \texttt{exec} with an all-bracketed rule instead of the \texttt{iterated}, to be able to match a multi-\texttt{cf}-edge target of \texttt{succ} multiple times (which is prevented in the \texttt{iterated} version by the isomorphy constraint locking the target after the first match).

This is the second version using exec instead, capable of handling multi edges:

  \begin{example}
    \begin{grgen}
rule T2exec
{
  pred:BB -e:cf-> succ:BB;
  negative {
    -e->;
    -:cf-> succ; // if succ has only one predecessor
  }
  
  modify { // then merge succ into this predecessor
    exec([copyToPred(pred, succ)] ;> delSucc(succ));
  }
}

rule copyToPred(pred:BB, succ:BB)
{
  succ -e:cf-> n:BB;
  
  modify {
    pred -:copy<e>-> n;
  }
}

rule delSucc(succ:BB)
{
  modify {
    delete(succ);
  }
}
    \end{grgen}
  \end{example}

Natural loops are so advantageous that one transforms irreducible graphs (which only occur by using wild gotos) into reducible ones, instead of bothering with them in the analyses and optimizations.
An irreducible graph can be made reducible by node splitting, which amounts to code duplication (in the program behind the control flow graph).
In a stuck situation after T1-T2 analysis, a \texttt{BB} node with multiple control flow predecessors is split into as many nodes as there are control flow predecessors, every one having the same control flow successors as the original node.
(Choosing the \texttt{cf} edges and \texttt{BB} nodes which yield the smallest amount of code duplication is another problem which we happily ignore here.)

  \begin{example}
We do the splitting by keeping the indeterministically chosen first cf edge, splitting off only further cf edges, replicating their common target.

    \begin{grgen}
rule split(succ:BB)
{
  pred:BB -first:cf-> succ;
  multiple {
    otherpred:BB -other:cf-> succ;
    
    modify {
      otherpred -newe:cf-> newsucc:copy<succ>;
      delete(other);
      exec(copyCfSuccFromTo(succ, newsucc));
    }
  }
  
  modify {
  }
}

rule copyCfSuccFromTo(pred:BB, newpred:BB)
{
  iterated {
    pred -e:cf-> succ:BB;
    
    modify {
      newpred -:copy<e>-> succ;
    }
  }
  
  modify {
  }
}
    \end{grgen}
  \end{example}

The examples given can be found in the \texttt{tests/mergeSplit/} directory including the control scripts and test graphs; you may add \texttt{debug} prefixes to the \texttt{xgrs} statements in the graph rewrite script files and call GrShell with e.g. \texttt{mergeSplit/split.grs} as argument from the \texttt{tests} directory to watch execution.

\subsubsection*{Node Replacement Grammars}\indexmain{node replacement grammar}\indexmainsee{edNCE}{node replacement grammar}
With node replacement grammars we mean edNCE grammars \cite{NodeReplacement}, which stands for edge label directed node controlled embedding. In this context free graph grammar formalism, every rule describes how a node with a nonterminal type is replaced by a subgraph containing terminal and nonterminal nodes and terminal edges. The nodes in the instantiated graph get connected to the nodes that were adjacent to the initial nonterminal node, by connection instructions which tell which edges of what direction and what type are to be created for which original edges of what direction and what type, going to a node of what type. 

This kind of grammars can be encoded in \GrG~by rules with a left hand side consisting of a node with a type denoting a nonterminal and iterateds matching the edges and opposite nodes it is connected to of interest; "of interest" amounts to the type and direction of the edges and the type of the opposite node. The right hand side deletes the original node (thus implicitely the incident edges), creates the replacement subgraph, and tells in the rewrite part of the iterateds what new edges of what directedness and type are to be created, from the newly created nodes to the nodes adjacent to the original node. (Multiple edges between two nodes are not allowed in the node replacement formalism, in case you want to handle them you've to use \texttt{exec} as shown in the merge/split example above.) 

The following example directly follows this encoding:

  \begin{example}
This is an example rule replacing a nonterminal node \texttt{n:NT} by a 3-clique.
For the outgoing \texttt{E1} edges of the original node, the new node \texttt{x} receives incoming \texttt{E2} edges.
And for incoming \texttt{E2} edges of the original node, the new nodes \texttt{y} and \texttt{z} receive edges of the same type, \texttt{y} with reversed direction and \texttt{z} of the exact dynamic subtype bearing the same values as the original edges.
    \begin{grgen}
rule example
{
  n:NT;

  iterated {
    n -:E1-> m:T;

    modify {
      x <-:E2- m;
    }
  }

  iterated {
    n <-e2:E2- m:T;   	

    modify {
      y -:E2-> m;
      z <-:copy<e2>- m;      
    }
  }
  
  modify {
    delete(n);
    x:T -- y:T -- z:T -- x; 
  }
}
    \end{grgen}
  \end{example}

As another example for node replacement grammars we encode the two rules needed for the generation of the completely connected graphs (cliques) in two \GrG~rules. The first replaces the nonterminal node by a new nonterminal node linked to a new terminal node, connecting both new nodes to all the nodes the original nonterminal node was adjacent to. The second replaces the nonterminal node by a terminal node, connecting the new terminal node to all the nodes the original nonterminal node was adjacent to. This "we want to preserve the original edges" can be handled more succintly and efficiently by retyping which we gladly use instead of the iteration.

  \begin{example}
    \begin{grgen}
rule cliqueStep
{
  nt:NT;
  
  iterated {
    nt -- neighbour:T;

    modify {
      t -- neighbour;
      nnt -- neighbour;
    }
  }
  
  modify {
    delete(nt);
    t:T -- nnt:NT;
  }
}

rule cliqueTerminal
{
  nt:NT;
  
  modify {
    :T<nt>;
  }
}
    \end{grgen}
  \end{example}

The examples can be found in the \texttt{tests/nodeReplacementGrammar} directory.

\subsubsection*{Copy structures}\label{subsub:copystructure}\indexmain{subgraph copying}\indexmainsee{copy structure}{subgraph copying}
Structures are copied in two passes, the first copying and collecting all nodes of interest, the second copying all edges of interest in between the nodes.

The first pass consists of covering the nodes of the structure one wants to copy with iterated subpatterns,
i.e. subpatterns which match from a root node on with iterateds along the incident edges into breadth,
employing a subpattern again on the node opposite to the root node to match into depth.
In the example we match the entire subgraph from a root node on, if one wants to copy a more constrained subgraph one can simple constrain the types, directions, and structures in the iterated subpattern covering the nodes.
The nodes are copied with the \texttt{copy} operators and a storagemap is filled, storing for every node copied its copy.

The second pass is started after the structure matching ended by executing the deferred execs which were issued for every node handled. 
Each \texttt{exec} copies all outgoing edges (one could process all incoming edges instead) of a node:
for each edge leaving the original node towards another original node a copy is created in between the copy of the original node and the copy of the other node.
The copies are looked up with the original nodes from the storage map (which fails for target nodes outside of the subgraph of interest).
Here too one could constrain the subgraph copied by filtering certain edges.
In case of undirected edges one would have to prevent that edges get copied twice (once for every incident node). This would require a visited flag for marking the already copied edges or a storage receiving them, queried in the edge copying pattern and set/filled in the edge copying rewrite part.

  \begin{example}
The example shows very generally how a subgraph reachable from a root node by incident edges can get copied, collecting and copying the nodes along a spanning tree from the root node on, then copying the edges in between the nodes in a second run afterwards. The edges get connected to the correct node copies via a mapping from the old to the new nodes remembered in a storage-map.
    \begin{grgen}
pattern CopySubgraph(root:Node, ref oldToNew:map<Node, Node>)
{
  iterated { // match spanning tree of graph from root on
    root <--> ch:Node;   	
    cs:CopySubgraph(ch, oldToNew);
    
    modify {
      cs();
    }
  }
  
  modify {
    newroot:copy<root>; // copy nodes
    eval { oldToNew.add(root, newroot); }
    exec( [CopyOutgoingEdge(root, oldToNew)] ); // deferred copy edges
  }
}

rule CopyOutgoingEdge(n:Node, ref oldToNew:map<Node, Node>)
{
  n -e:Edge-> m:Node;
  hom(n,m); // reflexive edges
  nn:Node<oldToNew[n]>; nm:Node<oldToNew[m]>;
  hom(nn,nm); // reflexive edges
    
  modify {
    nn -ee:copy<e>-> nm;
  }
}
    \end{grgen}
  \end{example}

The example can be found in the \texttt{tests/copyStructure} directory.
Without storagemaps one would have to pollute the graph model with helper edges linking the original to the copied nodes.

