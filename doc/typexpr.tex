\chapter{Types and Expressions}
\label{typeexpr}
In the following sections \emph{Ident} refers to an identifier of the graph model language (see Section~\ref{modelbb}) or the rule set language (see Section~\ref{rulebb}). \emph{TypeIdent} is an identifier of a node type or an edge type, \emph{NodeOrEdge} is an identifier of a node or an edge.

\section{Built-In Types}
\label{builtin}
Besides user-defined node types, edge types, and enumeration types, \GrG\ supports the built-in \indexed{primitive types}\indexmainsee{built-in types}{primitive types} in Table~\ref{builtintypes}.
The exact type format is \indexed{backend} specific. 
The \indexed{LGSPBackend} maps the \GrG\ primitive types to the corresponding C\# primitive types.
\begin{table}[htbp]
\begin{tabularx}{\linewidth}{|l|X|}\hline
	\texttt{\indexed{boolean}} & Covers the values \texttt{true} and \texttt{false} \\
	\texttt{\indexed{int}} & A signed integer with at least 32 bits \\
	\texttt{\indexed{float}}, \texttt{\indexed{double}} & A floating-point number with single precision or double precision respectively \\
	\texttt{\indexed{string}} & A character sequence of arbitrary length\\
        \texttt{\indexed{object}} & Contains a .NET object\\ \hline
\end{tabularx}
\caption{\GrG\ built-in primitive types}
\label{builtintypes}
\end{table}
Table~\ref{tabcasts} lists \GrG's implicit \indexed{type cast}s and the allowed explicit type casts.
Of course you are free to express an implicit type cast by an explicit type cast as well as ``cast'' a type to itself.

According to table~\ref{tabcasts} neither implicit nor explicit casts from {\tt int} to any \indexed{enum type} are allowed.
This is because the range of an enum type is very sparse in general.
For the same reason implicit and explicit casts between enum types are also forbidden.
Thus, enum values can only be assigned to attributes having the same enum type.
A cast of an enum value to a string value will return the declared name of the enum value.
A cast of an object value to a string value will return ``null'' or it will call the \texttt{toString()} method of the .NET object.
Be careful with assignments of objects: \GrG\ does not know your .NET type hierarchy and therefore it cannot check two objects for type compatibility.
\begin{table}[htbp]
  \centering
  \begin{tabular}[c]{|c|ccccccc|} \hline
    \backslashbox{to}{from} & \texttt{enum} & \texttt{boolean} & \texttt{int} & \texttt{float} & \texttt{double} & \texttt{string} & \texttt{object} \\ \hline
    \texttt{enum} & $=$/--- & & & & & & \\ 
    \texttt{boolean} & & $=$ & & & & & \\
    \texttt{int} & implicit & & $=$ & \texttt{(int)} & \texttt{(int)} & & \\
    \texttt{float} & implicit & & implicit & $=$ & \texttt{(float)} & & \\
    \texttt{double} &  implicit & & implicit & implicit & $=$ & & \\
    \texttt{string} & implicit & implicit & implicit & implicit & implicit & $=$ & implicit\\
    \texttt{object} & &  & & & & & $=$ \\\hline
  \end{tabular}
  \caption{\GrG\ type casts}
  \label{tabcasts}
\end{table}

\begin{example}
  \begin{itemize}
    \item Allowed:\\
	  \texttt{x.myfloat = x.myint; x.mydouble = (float) x.myint;\\ x.mystring = (string) x.mybool;}
    \item Forbidden:\\
      \texttt{x.myfloat = x.mydouble;} and \texttt{x.myint = (int) x.mybool;}\\
      \texttt{MyEnum1 = (MyEnum1Type) int;} and \texttt{MyEnum2 = (MyEnum2Type) MyEnum1;}
  where {\tt myenum1} and {\tt myenum2} are different enum types.

  \end{itemize}

\end{example}
\begin{note}
	Unlike an {\tt eval} part (which must not contain assignments to node or edge attributes) the declaration of an enum type can contain assignments of {\tt int} values to \indexed{enum item}s (see section~\ref{typedecl}).
	The reason is, that the range of an enum type is just defined in that context.
\end{note}

\section{Expressions}
%left to right, left associative?
\label{expressions}
\begin{rail}
  Expression: BoolExpr | IntExpr | FloatExpr | StringExpr | PrimaryExpr ;  
  BoolExpr: ((() | '!') PrimaryExpr) | (BoolExpr '?' BoolExpr ':' BoolExpr) | (BoolExpr BinBoolOperator BoolExpr) | (Expression CompareOperator Expression) | (TypeExpr CompareOperator TypeExpr);
\end{rail}\ixnterm{Expression}\ixnterm{BoolExpr}
The \texttt{!}\ operator negates a Boolean. 
Table~\ref{tabboolops} lists the binary operators for Boolean expressions. 
The \texttt{?}\ operator is a simple if-then-else: If the first \emph{BoolExpr} is evaluated to \texttt{true}, the operator returns the second \emph{BoolExpr}, otherwise it returns the third \emph{BoolExpr}.
The \emph{BinBoolOperator} is one of the operators in Table~\ref{tabboolops}.
The \emph{CompareOperator} is one of the following operators:
\[ \texttt{<} \;\;\;\;\; \texttt{<=} \;\;\;\;\; \texttt{==} \;\;\;\;\; \texttt{!=} \;\;\;\;\; \texttt{>=} \;\;\;\;\; \texttt{>} \]
These operators are supported by \texttt{int} types and \texttt{float}/\texttt{double} types (but by implicit casting they can also by used with all enum types).
\texttt{String} types, \texttt{boolean} types, and \texttt{object} types support only the \texttt{==} and the \texttt{!=} operators.
Table~\ref{compandtypes} describes the semantics of compare operators on \indexed{type expression}s.
\begin{table}[htbp]
  \centering
  \begin{tabularx}{\linewidth}{|l|X|} \hline
    \texttt{A == B} & True, iff $A$ and $B$ are identical. Different types in a type hierarchy are \emph{not} identical. \\
    \texttt{A != B} & True, iff $A$ and $B$ are not identical. \\
    \texttt{A <\ \ B} & True, iff $A$ is a supertype of $B$, but $A$ and $B$ are not identical. \\
    \texttt{A >\ \ B} & True, iff $A$ is a subtype of $B$, but $A$ and $B$ are not identical. \\
    \texttt{A <= B} & True, iff $A$ is a supertype of $B$ or $A$ and $B$ are identical. \\
    \texttt{A >= B} & True, iff $A$ is a subtype of $B$ or $A$ and $B$ are identical. \\ \hline
  \end{tabularx}
  \caption{Compare operators on type expressions}
  \label{compandtypes}
\end{table}
\begin{note}
  \texttt{A < B} corresponds to the direction of the arrow in an \indexed{UML class diagram}.
\end{note}

\begin{table}[htbp] 
  \centering
  %\begin{tabularx}{0.45\linewidth}{|ll|} \hline
  \begin{tabular}[c]{|lp{0.6\linewidth}|} \hline
    \begin{tabular}[c]{l} \texttt{\^} \end{tabular} & \begin{tabular}[c]{l} Logical XOR. True, iff either the first or the second \\ Boolean expression is true. \end{tabular} \\ \hline
    \begin{tabular}[c]{l} \texttt{\&\&} \\ \texttt{||} \end{tabular} & \begin{tabular}[c]{l} Logical AND and OR. Lazy evaluation. \end{tabular}\\ \hline
    \begin{tabular}[c]{l} \texttt{\&} \\ \texttt{|} \end{tabular} & \begin{tabular}[c]{l} Logical AND and OR. Strict evaluation. \end{tabular}\\ \hline
  \end{tabular}
  \caption{Binary Boolean operators, in ascending order of precedence}\indexmain{order of precedence}\indexmainsee{precedence}{order of precedence}
  \label{tabboolops}
\end{table}

\begin{rail}
  IntExpr: ((() | '+' | '-' | tilde) PrimaryExpr) | (BoolExpr '?' IntExpr ':' IntExpr) | (IntExpr BinIntOperator IntExpr);
\end{rail}\ixnterm{IntExpr}
The $\sim$ operator is the bitwise complement. 
That means every bit of an integer value will be flipped. 
The \texttt{?}\ operator is a simple if-then-else: If the \emph{BoolExpr} is evaluated to \texttt{true}, the operator returns the first \emph{IntExpr}, otherwise it returns the second \emph{IntExpr}. 
The \emph{BinIntOperator} is one of the operators in Table~\ref{tabbinops}.
\begin{table}[htbp] 
  \centering
  %\begin{tabularx}{0.45\linewidth}{|ll|} \hline
  \begin{tabular}[c]{|lp{0.6\linewidth}|} \hline
    \begin{tabular}[c]{l} \texttt{\^} \\ \texttt{\&} \\ \texttt{|} \end{tabular} & \begin{tabular}[c]{l} Bitwise XOR, AND and OR \end{tabular} \\ \hline
    \begin{tabular}[c]{l} \texttt{\mbox{<}\mbox{<}} \\ \texttt{\mbox{>}\mbox{>}} \\ \texttt{\mbox{>}\mbox{>}\mbox{>}} \end{tabular} & \begin{tabular}[c]{l} Bitwise shift left, bitwise shift right and \\ bitwise shift right preserving the sign \end{tabular}\\ \hline
    \begin{tabular}[c]{l} \texttt{+} \\ \texttt{-} \end{tabular} & \begin{tabular}[c]{l} Addition and subtraction \end{tabular}\\ \hline
    \begin{tabular}[c]{l} \texttt{*} \\ \texttt{/} \\ \texttt{\%} \end{tabular} & \begin{tabular}[c]{l}Multiplication, integer division, and modulo \end{tabular} \\ \hline
  \end{tabular}
  \caption{Binary integer operators, in ascending order of precedence}\indexmain{order of precedence}
  \label{tabbinops}
\end{table}

\begin{rail}  
  FloatExpr: ((() | '+' | '-') PrimaryExpr) | (BoolExpr '?' FloatExpr ':' FloatExpr) | (FloatExpr BinFloatOperator FloatExpr);
\end{rail}\ixnterm{FloatExpr}
The \texttt{?}\ operator is a simple if-then-else: If the \emph{BoolExpr} is evaluated to \texttt{true}, the operator returns the first \emph{FloatExpr}, otherwise it returns the second \emph{FloatExpr}.
The \emph{BinFloatOperator} is one of the operators in Table~\ref{tabfloatbinops}.
\begin{table}[htbp] 
  \centering
  %\begin{tabularx}{0.45\linewidth}{|ll|} \hline
  \begin{tabular}[c]{|ll|} \hline
    \begin{tabular}[c]{l} \texttt{+} \\ \texttt{-} \end{tabular} & \begin{tabular}[c]{l} Addition and subtraction \end{tabular}\\ \hline
    \begin{tabular}[c]{l} \texttt{*} \\ \texttt{/} \\ \texttt{\%} \end{tabular} & \begin{tabular}[c]{l}Multiplication, division and modulo \end{tabular} \\ \hline
  \end{tabular}
  \caption{Binary float operators, in ascending order of precedence}\indexmain{order of precedence}
  \label{tabfloatbinops}
\end{table}
\begin{note}
The \texttt{\%} operator on float values works analogous to the integer modulo operator. For instance \texttt{4.5 \% 2.3 == 2.2}.
\end{note}

\begin{rail}
  StringExpr: PrimaryExpr (SelectorExpr) | StringExpr '+' StringExpr;
\end{rail}\ixnterm{StringExpr}
The operator \texttt{+} concatenates two strings.
There are Several operations on strings available in method call notation, these are
 (examples are given for \texttt{n.str == "foo bar foo"} )
 
\begin{description}
\item[.length()] returns length of string, as int, \\
 e.g. \texttt{n.str.length()==11}
\item[.indexOf(strToSearchFor)] returns first position the string strToSearchFor appears at, as int, -1 if not found \\
 e.g. \texttt{n.str.indexOf("foo")==0}
\item[.lastIndexOf(strToSearchFor)] returns last position the string strToSearchFor appears at, as int, -1 if not found \\
 e.g. \texttt{n.str.lastIndexOf("foo")==8}
\item[.substring(startIndex, length)] returns substring of given length from startIndex on \\
 e.g. \texttt{n.str(4,3)=="bar"}
\item[.replace(startIndex, length, replaceStr)] return string with substring from startIndex of given length replaced by replaceStr, \\
 e.g. \texttt{n.str(4,3,"foo")=="foo foo foo"}
\end{description}

\begin{rail} 
  PrimaryExpr: '(' ('int' | 'float' | 'double' | 'string') ')' PrimaryExpr | '(' Expression ')' | (NodeOrEdge '.' Ident) | (EnumType '::' Ident) | ObjectIdent | Constant;
  Constant: Number | HexNumber | QuotedText | 'true' | 'false' | 'null';
\end{rail}\ixnterm{PrimaryExpr}\ixnterm{Constant}
\begin{description}
  \item[Number] Is an \texttt{int}, \texttt{float}, or \texttt{double} constant in decimal notation.
  \item[HexNumber] Is an \texttt{int} constant in hexadecimal notation starting with \texttt{0x}.
  \item[QuotedText] Is a string constant. It consists of a sequence of characters, enclosed by double quotes.
\end{description}

\section{Type Related Conditions}\indexmain{type expression}
\label{typeexpressions}

\begin{rail}
  TypeExpr: TypeIdent | 'typeof' '(' NodeOrEdge ')' ;
\end{rail}\ixkeyw{typeof}\ixnterm{TypeExpr}
A type expression identifies a type (and---in terms of matching---also its subtypes).
A type expression is either a type identifier itself or the type of a graph element.
The type expression \texttt{typeof(x)} stands for the type of the host graph element \texttt{x} is actually bound to.
\begin{example}
The following rule will add a reverse edge to a one-way street.
\begin{grgen}
rule oneway {
    a:Node -x:street-> y:Node;
    negative {
        y -:typeof(x)-> a;
    }
    replace {
        a -x-> y;
        y -:typeof(x)-> a;
    }
}
\end{grgen}
Remember that we have several subtypes of \texttt{street}. By the aid of the \texttt{typeof} operator, the reverse edge will be automatically typed correctly (the same type as the one-way edge). This behavior is not possible without the \texttt{typeof} operator.
\end{example}

\begin{rail}
  TypeConstraint: backslash '(' (TypeExpr + '+')  ')' ; 
\end{rail}\ixnterm{TypeConstraint}
A \indexed{type constraint} is used to exclude parts of the \indexed{type hierarchy}. The operator \texttt{+} is used to create a union of its operand types. So the following pattern statements are identical:\\
\begin{center}
\begin{tabular}[c]{|ll|ll|}\hline
\begin{tabular}{l}\texttt{x:T \char"5C\ (T1 + $\cdots$ + T$n$);}\\\end{tabular} &&& 
  \begin{tabular}{l}\texttt{x:T;} \\ \texttt{if \{!(\emph{typeof}(x) <= T1) \&\& $\cdots$} \\ \texttt{\ \ \ \ \&\& !(\emph{typeof}(x) <= T$n$)\}}\\\end{tabular} \\\hline
\end{tabular}
\end{center}
\begin{example}
\begin{tabularx}{\linewidth}{cX}
  \begin{tikzpicture}[baseline=(T.base)] \tt
    \begin{scope}[minimum size=0.5cm]
      \tikzstyle{every node}=[draw]
      \node (T)     at (1   ,4) {\texttt{T}};
      \node (T1)     at (1   ,3) {\texttt{T1}};
      \node (T2)     at (0   ,2) {\texttt{T2}};
      \node (T4)     at (0   ,1) {\texttt{T4}};
      \node (T3)     at (2   ,2) {\texttt{T3}};
    \end{scope}
    \draw[thick,-open triangle 45]  (T1) -> (T)  ;
    \draw[thick,-open triangle 45]  (T2) -> (T1)  ;
    \draw[thick,-open triangle 45]  (T3) -> (T1)  ;
    \draw[thick,-open triangle 45]  (T4) -> (T2)  ;
  \end{tikzpicture} &
  \parbox{\linewidth}{The expression \texttt{T\char"5C (T2+T3)} applied to the type hierarchy on the left side yields only the types \texttt{T} and \texttt{T1} as valid.}
\end{tabularx}
\end{example}

\section{Annotations}\indexmain{annotation}
\label{annotations}

Identifier \indexed{definition}s can be annotated by \indexedsee{pragma}{annotation}s. Annotations are key-value pairs.
\begin{rail}
  IdentDecl: Ident (() | '[' (Ident '=' Constant + ',') ']');
\end{rail}\ixnterm{IdentDecl}
Although you can use any key-value pairs between the brackets, only the identifier \ixkeyw{prio} has an effect so far.
\begin{table}[htbp]
\begin{tabularx}{\linewidth}{|lllX|} \hline
  \textbf{Key} & \textbf{Value Type} & \textbf{Applies to} & \textbf{Meaning} \\ \hline
  \texttt{prio} & int & node, edge & Changes the ranking of a graph element for \indexed{search plan}s. The default is \texttt{prio}=1000. Graph elements with high values are likely to appear prior to graph elements with low values in search plans.\\ \hline
\end{tabularx}
\caption{Annotations}
\label{tabannotations}
\end{table}
\begin{example}
We search the pattern \texttt{v:NodeTypeA -e:EdgeType-> w:NodeTypeB}. We have a host graph with about 100 nodes of \texttt{NodeTypeA}, 1,000 nodes of \texttt{NodeTypeB} and 10,000 edges of \texttt{EdgeType}. Furthermore we know that between each pair of \texttt{NodeTypeA} and \texttt{NodeTypeB} there exists at most one edge of \texttt{EdgeType}. \GrG\ can use this information to improve the initial search plan if we adjust the pattern like \texttt{v[prio=10000]:NodeTypeA -e[prio=5000]:EdgeType-> w:NodeTypeB}.
\end{example}

\section{TODO:Integrate}

\subsection{New attribute types set<K> and map<K,V>}
with K,V = boolean|int|float|double|string|enum
with several operations on them introduced.
(Implemented by C\#-Dictionary, i.e. Hashmaps)

\subsubsection{Changes in graph model}

Regarding sets:
An attribute of set type is declared like f.ex.
\begin{grgenlet} funnySet:set<int>; \end{grgenlet}
Optionally, you may initialize the sets by a set initializer
\begin{grgenlet} fancySet:set<string> = { "foo", "bar" }; \end{grgenlet}

Regarding maps:
An attribute of map type is declared like f.ex.
\begin{grgenlet} funnyMap:map<int, boolean>; \end{grgenlet}
Optionally, you may initialize the maps by a map initializer
\begin{grgenlet} fancyMap:map<string, int> = { "foo"->0, "bar"->1 }; \end{grgenlet}

Further on you may declare them constant.

\subsubsection{Changes in graph rewrite rules}

Regarding sets:

Assignment of sets
\begin{grgenlet} n.fancySet = m.fancySet; \end{grgenlet}
with value semantics like all the other attribute types, 
i.e. the set is copied, changes to one set don't show up in the other set.
The assignment is available only in the eval-block of the modify/replace-part, 
all the other/following operations are available in the if-block of the pattern, too.

Set constructors, analogous to set initializers, they represent a base set expression
\begin{grgenlet} e.g. n.fancySet = { "foo", "bar" }; \end{grgenlet}
but in contrast to set initializers, they may contain non constant values
\begin{grgenlet} e.g. n.fancySet = { n.strVal, m.strVal, n.strVal+m.strVal } \end{grgenlet}
The values must be all of the same type, if you wish implicit type casts
you have to prefix the set constructor by the set type
\begin{grgenlet} e.g. n.fancySet = set<string>{ 42, intVal, "fool" }; \end{grgenlet}
Set constructors create and return an anonymous set,
to be written to an attribute or to be operated on by some of the following set operations:

Operations on sets, forming expressions with sets
-Set union, intersection, difference as binary operators \begin{grgenlet} |,&,\ \end{grgenlet}
 e.g. \begin{grgenlet} n.fancySet \ { "la","le","lu" }; \end{grgenlet}
taking two sets of equal types and returning a new, computed set.
Priorities \begin{grgenlet} | < & < \ . \end{grgenlet} (higher priority binds stronger, so \begin{grgenlet} a&b\c|d == (a&(b\c))|d) \end{grgenlet}
-Set membership in, returning whether the set contains the given element, as boolean
 e.g. \begin{grgenlet} 42 in intSet \end{grgenlet}
%  or  \begin{grgenlet} m.boolValue = n.stringValue in { "foo", "bar" }; \end{grgenlet}

Methods on sets
.size() returns the number of elements in the set, as int
e.g. \begin{grgenlet}n.fancySet.size() \end{grgenlet}

Parameters of set type, can be used to hand some set in.
(The result of assigning to the input set is undefined.)
\begin{grgen}
test bla(var varSet:set<int>) 
{
	n:SomeType;
	if { n.foo in (varSet \ { n.a }); }
}
\end{grgen}

Regarding maps:

Assignment of maps
\begin{grgenlet} n.fancySet = m.fancySet; \end{grgenlet}
with value semantics like all the other attribute types, 
i.e. the map is copied, changes to one map don't show up in the other map.
The assignment is available only in the eval-block of the modify/replace-part, 
all the other/following operations are available in the if-block of the pattern, too.

Map constructors, analogous to map initializers, they represent a base map expression
 e.g. \begin{grgenlet} n.fancyMap = { "foo"->0, "bar"->1 }; \end{grgenlet}
but in contrast to map initializers, they may contain non constant values
 e.g. \begin{grgenlet} n.fancyMap = { n.strVal->m.intVal, m.strVal->n.intVal, (n.strVal+m.strVal)->(m.intVal+n.intVal) } \end{grgenlet}
The values must be all of the same type, if you wish implicit type casts 
you have to prefix the map constructor by the map type
 e.g. \begin{grgenlet} n.fancyMap = map<string,int>{ 42->42, intVal->strVal, "fool"->42 }; \end{grgenlet}
Map constructors create and return an anonymous map,
to be written to an attribute or to be operated on by some of the following map operations:

Operations on maps, forming expressions with maps
-Map union, intersection, difference as binary operators \begin{grgenlet} |,&,\ \end{grgenlet}
 e.g. \begin{grgenlet} n.fancyMap | { "la"->0,"le"->1,"lu"->2 }; \end{grgenlet}
taking two maps of equal types and returning a new, computed map.
Priorities \begin{grgenlet} | < & < \ . \end{grgenlet} (higher priority binds stronger, so  \begin{grgenlet}a&b\c|d == (a&(b\c))|d)\end{grgenlet}
Semantics of \begin{grgenlet} |,&,\ . \end{grgenlet} on maps:
 \begin{grgenlet}map1|map2\end{grgenlet} - new map with elements which are in at least one of the maps, with the value of map2 taking precedence
 \begin{grgenlet}map1&map2\end{grgenlet} - new map with elements which are in both maps, with the value of map2 taking precedence
 \begin{grgenlet}map1\map2\end{grgenlet} - new map with elements from map1 without the elements with a key contained in map2
-Additionally, \begin{grgenlet}map<K,V> \ set<K>\end{grgenlet} is allowed
 e.g. \begin{grgenlet}m.fancyMap \ { "hui","buh" } \end{grgenlet} 
 or  \begin{grgenlet}m.fancyMap = m.fancyMap \ n.fancySet \end{grgenlet}
-Map membership in, returning whether the map domain contains the given element, as boolean
 e.g. \begin{grgenlet}42 in intMap\end{grgenlet}  
 or  \begin{grgenlet}m.boolValue = n.stringValue in { "foo"->0, "bar"->1 }; \end{grgenlet}
-Map access \begin{grgenlet}map[key]\end{grgenlet} returning value in map saved under key (value in range mapped to by given key from domain)
 %e.g. \begin{grgenlet}n.intVal = m.fancyMap["bla"] + 42; \end{grgenlet}

Methods on maps
- .size() returns the number of elements in the map, as int
 e.g. n.fancyMap.size()
- .domain() returns the set of elements in the domain of the map
 e.g. n.fancyMap.domain(), result set<K>
- .range() returns the set of elements in the range of the map
 e.g. n.fancyMap.range(), result set<V>

Parameters of map type, can be used to hand some map in.
(The result of assigning to the input map is undefined.)
\begin{grgen}
test bla(var varMap:map<int,int>) 
{
	n:SomeType;
	if {  varMap[n.foo]==42; }
}
\end{grgen}

- Added visited flags
- Shorthand visited(el) for visited(el, 0)
 - Added nameof()-operator
    (e.g. "nameof(blub)" yields the name of the graph entity "blub";
     "nameof()" yields the name of the graph)
