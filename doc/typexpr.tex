\chapter{Types and Expressions}
\label{typeexpr}

In the following sections \emph{Ident} refers to an identifier of the graph model language (see section \ref{modelbb}) or the rule set language (see section \ref{rulebb}). \emph{TypeIdent} is an identifier of a node type or an edge type, \emph{NodeOrEdge} is an identifier of a node or an edge.

\section{Built-In Types}
\label{builtin}
Besides user-defined node types, edge types and enumeration types, GrGen supports the built-in \indexed{primitive types}\indexmainsee{built-in types}{primitive types} in table \ref{builtintypes}.
The exact type format is \indexed{backend} specific. 
The \indexed{LGSPBackend} maps the \GrG\ primitive types to the corresponding C\# primitive types.
\begin{table}[htbp]
\begin{tabularx}{\linewidth}{|l|X|}\hline
	\texttt{\indexed{boolean}} & Covers the values \texttt{true} and \texttt{false}. \\
	\texttt{\indexed{int}} & A signed integer with at least 32 bits. \\
	\texttt{\indexed{float}}, \texttt{\indexed{double}} & A floating-point number with single precision or double precision respectively. \\
	\texttt{\indexed{string}} & A character sequence of arbitrary length.\\ \hline
\end{tabularx}
\caption{\GrG\ built-in primitive types}
\label{builtintypes}
\end{table}
Table \ref{tabcasts} lists \GrG's implicit \indexed{type cast}s and the allowed explicit type casts. Of course you are free to express an implicit type cast by an explicit type cast as well as ``cast'' a type to itself, except for \indexed{enum type}s. The cast operator does never accept an enum type.
\begin{example}
  \texttt{myfloat = myint; mydouble = (float)myint; mystring = (string)mybool} is allowed, \\
  \texttt{myenum = (myenum)int; myfloat = mydouble; myint = (int)mybool} is forbidden.
\end{example}
\begin{table}[htbp]
  \centering
  \begin{tabular}[c]{|c|cccccc|} \hline
    \backslashbox{to}{from} & \texttt{enum} & \texttt{boolean} & \texttt{int} & \texttt{float} & \texttt{double} & \texttt{string}\\ \hline
    \texttt{enum} & $=$/--- & & & & & \\ 
    \texttt{boolean} & & $=$ & & & & \\
    \texttt{int} & \texttt{(int)} & & $=$ & \texttt{(int)} & \texttt{(int)} & \\
    \texttt{float} & \texttt{(float)} & & implicit & $=$ & \texttt{(float)} & \\
    \texttt{double} &  \texttt{(double)} & & implicit & implicit & $=$ & \\
    \texttt{string} & \texttt{(string)} & \texttt{(string)} & \texttt{(string)} & \texttt{(string)} & \texttt{(string)} & $=$ \\ \hline
  \end{tabular}
  \caption{\GrG\ type casts.}
  \label{tabcasts}
\end{table}

\section{Expressions}
\label{expressions}
\begin{rail}
  Expression: BoolExpr | IntExpr | FloatExpr | StringExpr | PrimaryExpr ;  
  BoolExpr: ((() | '!') PrimaryExpr) | (BoolExpr '?' BoolExpr ':' BoolExpr) | (BoolExpr BinBoolOperator BoolExpr) | (Expression CompareOperator Expression) | (TypeExpr CompareOperator TypeExpr);
\end{rail}
As in C \cite{Sch:1990:ANSIC}, \texttt{!}\ negates a Boolean. 
Table \ref{tabboolops} lists the binary operators for Boolean expressions. 
The \texttt{?}\ operator is a simple if-then-else: if the first \emph{BoolExpr} is evaluated to \texttt{true}, the operator returns the second \emph{BoolExpr}, otherwise it returns the third \emph{BoolExpr}.
The \emph{CompareOperator} is one of the following operators:
\[ \texttt{<} \;\;\;\;\; \texttt{<=} \;\;\;\;\; \texttt{==} \;\;\;\;\; \texttt{!=} \;\;\;\;\; \texttt{>=} \;\;\;\;\; \texttt{>} \]
These operators are supported by enum types, \texttt{int} types, and \texttt{float}/\texttt{double} types.
\texttt{String} types and \texttt{boolean} types support only the \texttt{==} and the \texttt{!=} operators.
Table \ref{compandtypes} describes the semantics of compare operators on \indexed{type expression}s.
The \emph{BinBoolOperator} is one of the operators in table \ref{tabboolops}.
\begin{table}[htbp]
  \centering
  \begin{tabularx}{\linewidth}{|l|X|} \hline
    \texttt{A == B} & True, iff $A$ and $B$ are identical. Different types in a type hierarchy are \emph{not} identical. \\
    \texttt{A != B} & True, iff $A$ and $B$ are not identical. \\
    \texttt{A < B} & True, iff $A$ is a supertype of $B$, but $A$ and $B$ are not identical. \\
    \texttt{A > B} & True, iff $A$ is a subtype of $B$, but $A$ and $B$ are not identical. \\
    \texttt{A <= B} & True, iff $A$ is a supertype of $B$ or $A$ and $B$ are identical. \\
    \texttt{A >= B} & True, iff $A$ is a subtype of $B$ or $A$ and $B$ are identical. \\ \hline
  \end{tabularx}
  \caption{Compare operators on type expressions}
  \label{compandtypes}
\end{table}
\begin{note}
  \texttt{A < B} corresponds to the direction of the arrow in an \indexed{UML class diagram}.
\end{note}

\begin{table}[htbp] 
  \centering
  %\begin{tabularx}{0.45\linewidth}{|ll|} \hline
  \begin{tabular}[c]{|lp{0.6\linewidth}|} \hline
    \begin{tabular}[c]{l} \texttt{\^} \end{tabular} & \begin{tabular}[c]{l} Logical XOR. True, iff either the first or the second \\ Boolean expression is true. \end{tabular} \\ \hline
    \begin{tabular}[c]{l} \texttt{\&\&} \\ \texttt{||} \end{tabular} & \begin{tabular}[c]{l} Logical AND and OR. Lazy evaluation. \end{tabular}\\ \hline
    \begin{tabular}[c]{l} \texttt{\&} \\ \texttt{|} \end{tabular} & \begin{tabular}[c]{l} Logical AND and OR. Strict evaluation. \end{tabular}\\ \hline
  \end{tabular}
  \caption{Binary Boolean operators, in ascending order of precedence}\indexmain{order of precedence}\indexmainsee{precedence}{order of precedence}
  \label{tabboolops}
\end{table}

\begin{rail}
  IntExpr: ((() | '+' | '-' | tilde) PrimaryExpr) | (BoolExpr '?' IntExpr ':' IntExpr) | (IntExpr BinIntOperator IntExpr);
\end{rail}
The $\sim$ operator is the bitwise complement. 
That means every bit of an integer value will be flipped. 
The \texttt{?}\ operator is a simple if-then-else: if the \emph{BoolExpr} is evaluated to \texttt{true}, the operator returns the first \emph{IntExpr}, otherwise it returns the second \emph{IntExpr}. 
The \emph{BinIntOperator} is one of the operators in table \ref{tabbinops}.
\begin{table}[htbp] 
  \centering
  %\begin{tabularx}{0.45\linewidth}{|ll|} \hline
  \begin{tabular}[c]{|lp{0.6\linewidth}|} \hline
    \begin{tabular}[c]{l} \texttt{\^} \\ \texttt{\&} \\ \texttt{|} \end{tabular} & \begin{tabular}[c]{l} Bitwise XOR, AND and OR \end{tabular} \\ \hline
    \begin{tabular}[c]{l} \texttt{\mbox{<}\mbox{<}} \\ \texttt{\mbox{>}\mbox{>}} \\ \texttt{\mbox{>}\mbox{>}\mbox{>}} \end{tabular} & \begin{tabular}[c]{l} Bitwise shift left, bitwise shift right and \\ bitwise shift right preserving the sign \end{tabular}\\ \hline
    \begin{tabular}[c]{l} \texttt{+} \\ \texttt{-} \end{tabular} & \begin{tabular}[c]{l} Addition and subtraction \end{tabular}\\ \hline
    \begin{tabular}[c]{l} \texttt{*} \\ \texttt{/} \\ \texttt{\%} \end{tabular} & \begin{tabular}[c]{l}Multiplication, integer division and modulo \end{tabular} \\ \hline
  \end{tabular}
  \caption{Binary integer operators, in ascending order of precedence}\indexmain{order of precedence}
  \label{tabbinops}
\end{table}

\begin{rail}  
  FloatExpr: ((() | '+' | '-') PrimaryExpr) | (BoolExpr '?' FloatExpr ':' FloatExpr) | (FloatExpr BinFloatOperator FloatExpr);
\end{rail} 
The \texttt{?}\ operator is a simple if-then-else: if the \emph{BoolExpr} is evaluated to \texttt{true}, the operator returns the first \emph{FloatExpr}, otherwise it returns the second \emph{FloatExpr}.
The \emph{BinFloatOperator} is one of the operators in table \ref{tabfloatbinops}.
\begin{table}[htbp] 
  \centering
  %\begin{tabularx}{0.45\linewidth}{|ll|} \hline
  \begin{tabular}[c]{|ll|} \hline
    \begin{tabular}[c]{l} \texttt{+} \\ \texttt{-} \end{tabular} & \begin{tabular}[c]{l} Addition and subtraction \end{tabular}\\ \hline
    \begin{tabular}[c]{l} \texttt{*} \\ \texttt{/} \\ \texttt{\%} \end{tabular} & \begin{tabular}[c]{l}Multiplication, division and modulo \end{tabular} \\ \hline
  \end{tabular}
  \caption{Binary float operators, in ascending order of precedence}\indexmain{order of precedence}
  \label{tabfloatbinops}
\end{table}
\begin{note}
The \texttt{\%} operator works analogous to the integer modulo operator. For instance \texttt{4.5 \% 2.3 == 2.2}. \GrG\ implements the Java semantics. 
\end{note}

\begin{rail}
  StringExpr: PrimaryExpr | StringExpr '+' StringExpr;
\end{rail}
The operator \texttt{+} concatenates two strings.

\begin{rail} 
  PrimaryExpr: '(' ('int' | 'float' | 'double' | 'string') ')' PrimaryExpr | '(' Expression ')' | (NodeOrEdge '.' Ident) | (EnumType '::' Ident) | Constant;
  Constant: Number | HexNumber | QuotedText | 'true' | 'false';
\end{rail}
\begin{description}
  \item[Number.] Is an \texttt{int}, \texttt{float} or \texttt{double} constant in decimal notation.
  \item[HexNumber.] Is an \texttt{int} constant in hexadecimal notation starting with \texttt{0x}.
  \item[QuotedText.] Is a string constant. It consists of a sequence of characters, enclosed by double quotes.
\end{description}

\section{Type Related Conditions}\indexmain{type expression}
\label{typeexpressions}

\begin{rail}
  TypeExpr: TypeIdent | 'typeof' '(' NodeOrEdge ')' ;
\end{rail}\indexmain{\texttt{typeof}}
A type expression identifies a type (and---in terms of matching---also its subtypes). A type expression is either a type identifier itself or the type of a graph element.
\begin{example}
The following rule will add a reverse edge to a one-way street.
\begin{grgen}
rule oneway {
    pattern {
        a:Node -x:street-> y:Node;
        negative{
            y -:typeof(x)-> a;
        }
    } 
    replace {
        a -x-> y;
        y -:typeof(x)-> a;
    }
}
\end{grgen}
Remember that we have several subtypes of \texttt{street}. By the aid of the \texttt{typeof} operator, the reverse edge will be automatically typed correctly (the same type as the one-way edge). This behavior is not possible without the \texttt{typeof} operator.
\end{example}

\begin{rail}
  TypeConstraint: backslash '(' (TypeExpr + '+')  ')' ; 
\end{rail}
A \indexed{type constraint} is used to exclude parts of the \indexed{type hierarchy}. The operator \texttt{+} is used to create a union of its operand types. So the following pattern statements are identical:\\
\begin{center}
\begin{tabular}[c]{lll}
\begin{tabular}{|l|}\hline\texttt{x:T \char"5C\ (T1 + T2);}\\\hline\end{tabular} && 
  \begin{tabular}{|l|}\hline\texttt{x:T;} \\ \texttt{if \{!(\emph{typeof}(x) == T1) \&\& !(\emph{typeof}(x) == T2)\}}\\\hline\end{tabular}
\end{tabular}
\end{center}
\begin{example}
\begin{tabularx}{\linewidth}{cX}
  \begin{tikzpicture}[baseline=(A.base)] \tt
    \begin{scope}[minimum size=0.5cm]
      \tikzstyle{every node}=[draw]
      \node (A)     at (1   ,4) {A};
      \node (B)     at (1   ,3) {B};
      \node (C)     at (0   ,2) {C};
      \node (D)     at (0   ,1) {D};
      \node (E)     at (2   ,2) {E};
    \end{scope}
    \draw[thick,-open triangle 45]  (B) -> (A)  ;
    \draw[thick,-open triangle 45]  (C) -> (B)  ;
    \draw[thick,-open triangle 45]  (E) -> (B)  ;
    \draw[thick,-open triangle 45]  (D) -> (C)  ;
  \end{tikzpicture} &
  \parbox{\linewidth}{The expression \texttt{A\char"5C (C+E)} applied to the type hierarchy on the left side yields only the types $A$ and $B$ as valid.}
\end{tabularx}
\end{example}

\section{Annotations}\indexmain{annotation}
\label{annotations}

Identifier \indexed{definition}s can be annotated by \indexedsee{pragma}{annotation}s. Annotations are key-value pairs.
\begin{rail}
  IdentDecl: Ident (() | '[' (Ident '=' Constant + ',') ']');
\end{rail}
Although you can use any key-value pairs between the brackets, only the identifier \indexed{\texttt{prio}} has an effect so far.
\begin{table}[htbp]
\begin{tabularx}{\linewidth}{|lllX|} \hline
  \textbf{Key} & \textbf{Value Type} & \textbf{Applies to} & \textbf{Meaning} \\ \hline
  \texttt{prio} & int & node, edge & Changes the ranking of a graph element for \indexed{search plan}s. The default is \texttt{prio}=1000. Graph elements with high values are likely to appear prior to graph elements with low values in search plans.\\ \hline
\end{tabularx}
\caption{Annotations}
\label{tabannotations}
\end{table}
\begin{example}
We search the pattern \texttt{v:NodeTypeA -e:EdgeType-> w:NodeTypeB}. We have a host graph with about 100 nodes of \texttt{NodeTypeA}, 1,000 nodes of \texttt{NodeTypeB} and 10,000 edges of \texttt{EdgeType}. Furthermore we know that between each pair of \texttt{NodeTypeA} and \texttt{NodeTypeB} there exists at most one edge of \texttt{EdgeType}. \GrG\ can use this information to improve the initial search plan, if we adjust the pattern like \texttt{v[prio=10000]:NodeTypeA -e[prio=5000]:EdgeType-> w:NodeTypeB}.
\end{example}

