\chapter{Container and Graph Types}
\label{cha:container}


%%%%%%%%%%%%%%%%%%%%%%%%%%%%%%%%%%%%%%%%%%%%%%%%%%%%%%%%%%%%%%%%%%%%%%%%%%%%%%%%%%%%%%%%%%%%%%%%
\section{Built-In Types}
\label{sec:builtintypes}
Besides the types already introduced, \GrG\ supports the built-in \indexed{generic types}\indexmainsee{built-in generic types}{generic types} in Table~\ref{builtingenerictypes}.
The exact type format is \indexed{backend} specific.
The \indexed{LGSPBackend} maps the \GrG\ the generic types to generic C\#-Dictionaries (i.e. hashmaps) or C\#-Lists (misnamed dynamic arrays) of their corresponding primitive types, with \texttt{de.unika.ipd.grGen.libGr.SetValueType} as target type for sets, only used with the value \texttt{null}.

\begin{table}[htbp]
\begin{tabularx}{\linewidth}{|l|X|}
	\hline
	\texttt{\indexed{set}<T>} & A (mathematical) set of type \texttt{T}, where \texttt{T} may be an enumeration type or one of the primitive types from above; it may even be a node or edge type, then we speak of storages. \\
	\texttt{\indexed{map}<S,T>} & A (mathematical) map from type \texttt{S} to type \texttt{T}, where \texttt{S} and \texttt{T} may be enumeration types or one of the primitive types from above; it may even be a node or edge type, then we speak of storages. \\
	\texttt{\indexed{array}<T>} & An array of type \texttt{T}, where \texttt{T} may be an enumeration type or one of the primitive types from above; it may even be a node or edge type, then we speak of storages. Shares some similarities with \texttt{map<int,T>}. \\
	\hline
	\texttt{\indexed{graph}} & A subgraph of the host graph\\
	\hline
\end{tabularx}
\caption{\GrG\ built-in generic types}
\label{builtingenerictypes}
\end{table}

The type \texttt{graph} is not availble for general multi-graph processing but for storing subgraphs of the current host graph.
The only operations supported on it are equality and inequality, i.e. graph isomorphy checking 
(they are considered equal if they can be mapped onto each other showing the same structure and attribute values).

%%%%%%%%%%%%%%%%%%%%%%%%%%%%%%%%%%%%%%%%%%%%%%%%%%%%%%%%%%%%%%%%%%%%%%%%%%%%%%%%%%%%%%%%%%%%%%%%
\section{Expressions}\indexmain{expression}\label{sub:expr}

\GrG~supports numerous operations on the entities of the types introduced above, which are organized into left associative expressions.
In the following they will be explained with their semantics and relative priorities one type after another in the order of the rail diagram below.

\begin{rail}
  Expression: RelationalExpr | SetExpr | MapExpr | ArrayExpr | PrimaryExpr;
\end{rail}\ixnterm{Expression}


%%%%%%%%%%%%%%%%%%%%%%%%%%%%%%%%%%%%%%%%%%%%%%%%%%%%%%%%%%%%%%%%%%%%%%%%%%%%%%%%%%%%%%%%%%%%%%%%
\section{Relational Expressions}

The relational expressions compare enitites of different kinds, mapping them to the type boolean.
They were alreay introduced in \ref{sec:relational}, here we extend them with set, map, array and graph types.

For set/map/array expressions, table~\ref{compandsetmap} describes the semantics of the compare operators.
Some set A is a subset of B iff all elements in A are contained in B, too.
A map A is a submap of B iff all key-value pairs of A are contained in B, too. If they have a key in common but map to a different value, they count as not identical.
An array A is a subarray of B iff it is smaller or equal in size and the values at each common index are identical.
\begin{table}[htbp]
  \centering
  \begin{tabularx}{\linewidth}{|l|X|} \hline
    \texttt{A == B} & True, iff $A$ and $B$ are identical. \\
    \texttt{A != B} & True, iff $A$ and $B$ are not identical. \\
    \texttt{A <\ \ B} & True, iff $A$ is a subset/map/array of $B$, but $A$ and $B$ are not identical. \\
    \texttt{A >\ \ B} & True, iff $A$ is a superset/map/array of $B$, but $A$ and $B$ are not identical. \\
    \texttt{A <= B} & True, iff $A$ is a subset/map/array of $B$ or $A$ and $B$ are identical. \\
    \texttt{A >= B} & True, iff $A$ is a superset/map/array of $B$ or $A$ and $B$ are identical. \\ \hline
  \end{tabularx}
  \caption{Compare operators on set/map/array expressions}
  \label{compandsetmap}
\end{table}

\begin{table}[htbp]
  \centering
  \begin{tabularx}{\linewidth}{|l|X|} \hline
    \texttt{A == B} & True, iff $A$ is isomorphic to $B$. \\
    \texttt{A != B} & True, iff $A$ is not isomorphic to $B$. \\
    \texttt{A \textasciitilde\textasciitilde{ } B} & True, iff $A$ is structurally the same as $B$ but maybe different regarding the attributes. \\ \hline
  \end{tabularx}
  \caption{Compare operators on graph expressions}
  \label{compandgraph}
\end{table}

\texttt{Graph} types support only the \texttt{==}, the \texttt{!=}, and the \texttt{\textasciitilde\textasciitilde} operators;
on (sub)graph types they tell whether the (sub)graphs are isomorphic to each other (\indexed{isomorphy checking}/\indexed{graph isomorphy checking}) or not, including the attributes, or whether the (sub)graphs are isomorphic disregarding the attributes.


%%%%%%%%%%%%%%%%%%%%%%%%%%%%%%%%%%%%%%%%%%%%%%%%%%%%%%%%%%%%%%%%%%%%%%%%%%%%%%%%%%%%%%%%%%%%%%%%
\section{Set Expression}\label{sec:setexpr}

Set expressions consist of the known mathematical set operations, plus some operations in method call notation.

\begin{rail}
  SetExpr: Expr 'in' SetExpr | SetExpr (backslash | ampersand | '|') SetExpr | PrimaryExpr (MethodSelector)?;
\end{rail}\ixnterm{SetExpr}\ixkeyw{in}

\begin{table}[htbp]
  \centering
  %\begin{tabularx}{0.45\linewidth}{|ll|} \hline
  \begin{tabular}[c]{|ll|} \hline
    \begin{tabular}[c]{l} \verb#|# \end{tabular} & \begin{tabular}[c]{l}Set union (contained in resulting set as soon as contained in one of the sets)\end{tabular}\\ \hline
    \begin{tabular}[c]{l} \verb#&# \end{tabular} & \begin{tabular}[c]{l}Set intersection (contained in resulting set only if contained in both of the sets)\end{tabular} \\ \hline
    \begin{tabular}[c]{l} \verb#\# \end{tabular} & \begin{tabular}[c]{l}Set difference (contained in resulting set iff contained in left but not right set)\end{tabular} \\ \hline
  \end{tabular}
  \caption{Binary set operators, in ascending order of precedence}\indexmain{order of precedence}
  \label{tabsetbinops}
\end{table}

The binary set operators require the left and right operands to be of identical type \verb#set<T>#.
The operator \texttt{x in s} denotes set membership $x \in s$, returning whether the set contains the given element, as \texttt{boolean}.
Furthermore there are two operations on sets available in method call notation (MethodSelector):

\begin{description}
\item[\texttt{.size()}] returns the number of elements in the set, as \texttt{int}
\item[\texttt{.peek(num)}] returns the element which comes at position \texttt{num:int} in the sequence of enumeration, as \texttt{T} for \verb#set<T>#; the higher the number, the longer retrieval takes
\end{description}

\begin{note}
To add a value to a set you may use set union with a single valued set constructor,
to remove a value from a set you may use set difference with a single valued set constructor (for set constructors cf. \ref{sec:primexpr}).
\begin{grgen}
s | { "foo" }
s \ { n.a }
\end{grgen}
Used in this way they get internally optimized to the imperative set addition \texttt{s.add(x)} and removal \texttt{s.rem(x)} methods available in the \texttt{eval} block and the XGRS.
\end{note}


%%%%%%%%%%%%%%%%%%%%%%%%%%%%%%%%%%%%%%%%%%%%%%%%%%%%%%%%%%%%%%%%%%%%%%%%%%%%%%%%%%%%%%%%%%%%%%%%
\section{Map Expression} \label{sec:mapexpr}

Map expressions consist of the known mathematical set operations extended to maps, and map value lookup, plus some operations in method call notation.

\begin{rail}
  MapExpr: Expr 'in' MapExpr | MapExpr '[' Expr ']' | MapExpr (backslash | ampersand | '|') MapExpr | PrimaryExpr (MethodSelector)?;
\end{rail}\ixnterm{MapExpr}\ixkeyw{in}

\begin{table}[htbp]
  \centering
  %\begin{tabularx}{0.45\linewidth}{|ll|} \hline
  \begin{tabular}[c]{|ll|} \hline
    \begin{tabular}[c]{l} \verb#|# \end{tabular} & \begin{tabular}[c]{l}Map union: returns new map with elements which are in at least one of the maps,\\ with the value of map2 taking precedence\end{tabular}\\ \hline
    \begin{tabular}[c]{l} \verb#&# \end{tabular} & \begin{tabular}[c]{l}Map intersection: returns new map with elements which are in both maps,\\ with the value of map1 taking precedence\end{tabular} \\ \hline
    \begin{tabular}[c]{l} \verb#\# \end{tabular} & \begin{tabular}[c]{l}Map difference: returns new map with elements from map1\\ without the elements with a key contained in map2\end{tabular}\\ \hline
  \end{tabular}
  \caption{Binary map operators, in ascending order of precedence}\indexmain{order of precedence}
  \label{tabmapbinops}
\end{table}

The binary map operators require the left and right operands to be of identical type \verb#map<S,T>#,
with one exception for map difference, this operator accepts for a left operand of type \verb#map<S,T># a right operand of type \verb#set<S>#, too.
The operator \texttt{x in m} denotes map domain membership $x \in dom(m)$, returning whether the domain of the map contains the given element, as \texttt{boolean}.
The operator \texttt{m[x]} denotes map lookup, i.e. it returns the value \texttt{y} which is stored in the map \texttt{m} for the value \texttt{x} (domain value \texttt{x} is mapped by the mapping \texttt{m} to range value \texttt{y}). The value \texttt{x} \emph{must} be in the map, i.e. \texttt{x in m} must hold.
There are several operations on maps available in method call notation (MethodSelector), these are:

\begin{description}
\item[\texttt{.size()}] returns the number of elements in the map, as \texttt{int}
\item[\texttt{.domain()}] returns the set of elements in the domain of the map, as \verb#set<S># for \verb#map<S,T>#
\item[\texttt{.range()}] returns the set of elements in the range of the map, as \verb#set<T># for \verb#map<S,T>#
\item[\texttt{.peek(num)}] returns the key of the element which comes at position \texttt{num:int} in the sequence of enumeration, as \texttt{S} for \verb#map<S,T>#; the higher the number, the longer retrieval takes
\end{description}

\begin{note}
To add a key,value-pair to a map you may use map union with a single valued map constructor,
to remove a value from a map you may use map difference with a single valued set or map constructor (for map constructors cf. \ref{sec:primexpr}).
\begin{grgen}
m | { "foo" -> 42 }
m \ { n.a -> n.b } or m \ { n.a }
\end{grgen}
Used in this way they get internally optimized to the imperative map addition \texttt{s.add(key,value)} and removal \texttt{s.rem(key)} methods available in the \texttt{eval} block and the XGRS.
\end{note}


%%%%%%%%%%%%%%%%%%%%%%%%%%%%%%%%%%%%%%%%%%%%%%%%%%%%%%%%%%%%%%%%%%%%%%%%%%%%%%%%%%%%%%%%%%%%%%%%
\section{Array Expression} \label{sec:arrayexpr}

Array expressions consist of array membership checking, array value lookup, and array concatenation, plus some operations in method call notation.

\begin{rail}
  ArrayExpr: Expr 'in' ArrayExpr | ArrayExpr '[' Expr ']' | ArrayExpr '+' ArrayExpr | PrimaryExpr (MethodSelector)?;
\end{rail}\ixnterm{ArrayExpr}\ixkeyw{in}

The operator \texttt{x in a} denotes array value membership, returning whether the array contains the given element, as \texttt{boolean}.
The operator \texttt{a[x]} denotes array lookup, i.e. it returns the value \texttt{y} which is stored in the array \texttt{a} at the index \texttt{x}.
The index \texttt{x} \emph{must} be a valid array index.
The operator \texttt{x + y} denotes array concatenation, which requires the left and right operands to be of identical type \verb#array<T>#.
There are several operations on arrays available in method call notation (MethodSelector), these are:

\begin{description}
\item[\texttt{.size()}] returns the number of elements in the array, as \texttt{int}
\item[\texttt{.indexOf(valueToSearchFor)}] returns first position \texttt{valueToSearchFor:T} appears at, as \texttt{int}, or -1 if not found
\item[\texttt{.lastIndexOf(valueToSearchFor)}] returns last position \texttt{valueToSearchFor:T} appears at, as \texttt{int}, or -1 if not found
\item[\texttt{.subarray(startIndex, length)}] returns subarray of given \texttt{length:int} from \texttt{startIndex:int} on
\item[\texttt{.peek(num)}] returns the value stored in the array at position \texttt{num:int} in the sequence of enumeration, is equivalent to (and implemented by) \texttt{a[num])}; retrieval occurs in constant time.
\end{description}

\begin{note}
In contrast to sets and maps which can be handled well by expressions, arrays are meant to be modified by imperative \texttt{eval} block and XGRS statements like array addition \texttt{a.add(value)}/\texttt{a.add(value, index)} and removal \texttt{s.rem()}/\texttt{s.rem(index)}.
For declarative array construction please have a look at \ref{sec:primexpr}.
\end{note}


%%%%%%%%%%%%%%%%%%%%%%%%%%%%%%%%%%%%%%%%%%%%%%%%%%%%%%%%%%%%%%%%%%%%%%%%%%%%%%%%%%%%%%%%%%%%%%%%
\section{Primary expressions}\label{sec:containerprimexpr}

After we've seen the all the ways to combine expressions, finally we'll have a look at the atoms the expressions are built of.

A function call as already inroduced in \ref{sec:primexpr} employs an external (attribute evaluation) function (cf. \ref{sub:extfct}) or one of the following built-in functions:

\begin{description}
\item[\texttt{incoming(.)}] returns a set of the edges which are incoming to the node given as argument value.
\item[\texttt{incoming(.,.)}] as above, but only edges of the edge type given as second argument are contained.
\item[\texttt{incoming(.,.,.)}] as above, but only if the edges originate from a node of the node type given as third argument.
\item[\texttt{outgoing(.)}] returns a set of the edges which are outgoing from the node given as argument value.
\item[\texttt{outgoing(.,.)}] as above, but only edges of the edge type given as second argument are contained.
\item[\texttt{outgoing(.,.,.)}] as above, but only if the edges lead to a node of the node type given as third argument.
\end{description}

Besides the \emph{Constant} already introduced as a part of the \emph{Literal}, cf. \ref{literaldef}, the literals \emph{SetConstructor}, \emph{MapConstructor}, \emph{ArrayConstructor} are supported as well.

\begin{rail}
  SetConstructor: ('set' '<' Type '>')? \\ lbrace ( Expression*',' ) rbrace ;
  MapConstructor: ('map' '<' Type ',' Type '>')? \\ lbrace ( (Expression '->' Expression)*',' ) rbrace ;
  ArrayConstructor: ('array' '<' Type '>')? \\ '[' ( Expression*',' ) ']' ;
\end{rail}\ixnterm{SetConstructor}\ixnterm{MapConstructor}\ixnterm{ArrayConstructor}

The set/map/array constructors are constant if only primitive type literals, enum literals, or constant expressions are used; this is required for set/map/array initializations in the model.
They are non-constant if they contain nodes/edges/or member accesses, which is the common case if used in rules.
If the type of the set/map/array is given before the constructor, the elements given in the type constructor are casted to the specified member types if needed and possible.
Without the type prefix all elements given in the constructor must be of the same type.

\begin{example}
Some examples of literals:
\begin{grgen}
{ "foo", "bar" } // a constant set<string> constructor
map<string,int>{ (n.strVal+m.strVal)->(m.intVal+n.intVal), intVal->strVal, "fool"->42 } // a non-constant map constructor with type prefix
[ 1,2,3 ] // a constant array<int> constructor
\end{grgen}
\end{example}
