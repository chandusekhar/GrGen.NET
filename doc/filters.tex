\chapter{Filtering and Sorting of Matches}\indexmain{filters}\label{sub:filters}

Filters are used to process the matches list of a rule all application (including the one-element matches list of a single rule application) after all matches were found, but before they are rewritten.
They allow you to follow only the \emph{most promising matches} during a search task.

You may implement filters in the form of \emph{filter functions} on your own.
Alternatively, you may declare certain filters at their rules, they are then \emph{auto-generated} for you.
Some filters are \emph{auto-supplied} and may just be used.
All filters are used by \emph{filter calls} from the sequences (normally together with a rule all application).


%%%%%%%%%%%%%%%%%%%%%%%%%%%%%%%%%%%%%%%%%%%%%%%%%%%%%%%%%%%%%%%%%%%%%%%%%%%%%%%%%%%%%%%%%%%%%%%%
\section{Filter Functions}

Filter functions need to be \emph{defined}, i.e. declared and implemented before they can be \emph{used}.

\begin{rail}
	FilterFunctionDefinition: 'filter' FilterIdent '<' RuleIdent '>' (Parameters | ()) \\ lbrace (Statement+) rbrace;
\end{rail}\ixnterm{FilterFunctionDefinition}\ixkeyw{filter}

A \indexed{filter function definition} is given in between the rules as a global construct.
It specifies the name of the filter and the parameters supported, furthermore it specifies which rule the filter applies to, and it supplies a list of statements in the body. 

The restrictions of a function body apply here, i.e. you are not allowed to manipulate the graph.
In contrast to the function body is a \texttt{this} variable predefined, which gives access to the matches array to filter, of type \texttt{array<match<r>>}, where \texttt{r} denotes the name of the rule the filter is defined for.
All the operations known for variables of array type (cf. \ref{sec:arrayexpr}) are available.

\begin{example}
The following filter \texttt{ff} for the rule \texttt{foo} clones the last match in the array of matches, and adds it to the array.
So the last match would be rewritten twice (which is fine for \texttt{foo} that only creates a node, but take care when deleting or retyping nodes).
But in the following the first entry and the last entry are deleted (again); the first in an efficient in-place way by assigning \texttt{null}.
\begin{grgen}
rule foo
{
	n:Node;
	modify{
		nn:Node;
	}
}

filter ff<foo>
{
	this.add(copy(this.peek()));
	this[0] = null; // removes first entry in matches array, efficiently
	this.rem();
}
\end{grgen}
\end{example}

\indexmain{copy}The \indexed{match type} \texttt{match<r>} itself provides member access in dot notation (reading and writing), and a \indexed{\texttt{copy}} operation to get a clone (for insertion into the matches array).

\begin{example}
The following rule \texttt{incidency} yields for each node \texttt{n} matched the number of incident edges to a def variable \texttt{i} and assigns it in the eval on to an attribute \texttt{j}.
The filter \texttt{filterMultiply} modifies the def variable \texttt{i} in the matches with the factor \texttt{f} handed in as filter parameter.
\begin{grgen}
rule incidency
{
	n:N;
	def var i:int;
	yield { yield i = incident(n).size(); }
	
	modify{
		eval { n.j = i; }
	}
}

filter filterMultiply<incidency>(var f:int)
{
	for(m:match<incidency> in this)
	{
		m.i = m.i * f;
	}
}
\end{grgen}
The following call triples the count of incident edges that is finally written to the attribute \texttt{j} for each node \texttt{n} matched by \texttt{incidency}:
\begin{grshell}
  exec [incidency \ filterMultiply(3)]
\end{grshell}
\end{example}

You may declare external filters you must implement then in a C\# accompanying file, see \ref{sub:extflt} for more on that.
With filter functions and external filter functions, you may implement matches-list modifications exceeding the available pre-implemented functionality.


%%%%%%%%%%%%%%%%%%%%%%%%%%%%%%%%%%%%%%%%%%%%%%%%%%%%%%%%%%%%%%%%%%%%%%%%%%%%%%%%%%%%%%%%%%%%%%%%
\section{Auto-Generated Filters}

Auto-generated filters need to be \emph{declared} at their rule before they can be \emph{used}, they are implemented by \GrG.

\begin{rail}
  FiltersDecl: ( backslash (FilterDecl + ',') )?;
  FilterDecl: ( FilterIdent '<' VariableIdent '>' | 'auto' );
\end{rail}\ixnterm{FiltersDecl}\ixnterm{FilterDecl}\ixkeyw{auto}

The \indexed{auto-generated filters} must be declared at the end of a rule definition (cf. \ref{ruledecls}).
With exception of the \texttt{auto} filter for removal of automorphic matches they have to specify the name of a \texttt{def var} variable contained in the pattern, of integer, floating point, or string type; you \texttt{yield} a value computed from the elements and their attributes to it, per matched pattern.
They filter based on certain conditions regarding that variable; filtering may mean to reorder a matches list alongside the chosen def variable.

\begin{description}
\item[\texttt{orderAscendingBy<v>}] orders the matches list ascendingly (from lowest ot highest value according to the \verb#<# operator), alongside the \texttt{v} contained in each match.
\item[\texttt{orderDescendingBy<v>}] orders the matches list descendingly, alongside the \texttt{v} contained in each match.
%\item[\texttt{groupBy<v>}] ensures matches of equal \texttt{v} values are neighbours or only separated by matches of equal \texttt{v} values. Ordering subsumes this. (Grouping could be applied on types that only support the \verb#==# operator, though.)
\item[\texttt{keepSameAsFirst<v>}] filters away all matches with \texttt{v} values that are not equal to the \texttt{v} value of the first match. May be used to ensure all matches of an indeterministically chosen value are rewritten, but typically you want to order the matches list before.
\item[\texttt{keepSameAsLast<v>}] filters away all matches with \texttt{v} values that are not equal to the \texttt{v} value of the last match.
\item[\texttt{keepOneForEach<v>}] filters away all matches with duplicate \texttt{v} values, i.e. only one (prototypical) match is kept per \texttt{v} value. The list must have been grouped or ordered before, otherwise the result is undefined.
\end{description}
 
Besides those variable-based filters you may use the automorphic matches filter to purge \indexed{symmetric matches}.
In order to do so specify the special filter named \texttt{auto} at the rule declaration, and use it with the same name at the an application of that action (with the known backslash syntax).
The filter is removing matches due to an \indexed{automorphic} pattern, matches which are covering the same spot in the host graph with a permutation of the nodes, the edges, or subpatterns of the same type at the same level.
Other nested pattern constructs which are structurally identical are not recognized to be identical when they appear as commuted subparts; but they are so when they are factored into a subpattern which is instantiated multiple times.
It is highly recommended to use this \indexed{symmetry reduction} technique when building state spaces including isomorphic state collapsing, as puring the matches which lead to isomorphic states early saves expensive graph comparisons -- and often it gives the semantics you are interested in anyway.


%%%%%%%%%%%%%%%%%%%%%%%%%%%%%%%%%%%%%%%%%%%%%%%%%%%%%%%%%%%%%%%%%%%%%%%%%%%%%%%%%%%%%%%%%%%%%%%%
\section{Auto-Supplied Filters}

A few filters (that do not need information from a specific match) are auto-supplied, they can be used without implementation and even without declaration. They allow to keep (and thus remove) a certain number of matches from the beginning or the end of the matches list.

The following auto-supplied filters may be used directly:
\begin{description}
\item[\texttt{keepFirst(count)}] keeps the first \texttt{count} matches from the begin of the matches list, \texttt{count} must be an integer number.
\item[\texttt{keepLast(count) }] keeps the first \texttt{count} matches from the end of the matches list, \texttt{count} must be an integer number.
\item[\texttt{keepFirstFraction(fraction)}] keeps the \texttt{fraction} of the matches from the begin of the matches list, \texttt{fraction} must be a floating point number in between \verb#0.0# and \verb#1.0#.
\item[\texttt{keepLastFraction(fraction)}] keeps the \texttt{fraction} of the matches from the end of the matches list, \texttt{fraction} must be a floating point number in between \verb#0.0# and \verb#1.0#.
\end{description}

%%%%%%%%%%%%%%%%%%%%%%%%%%%%%%%%%%%%%%%%%%%%%%%%%%%%%%%%%%%%%%%%%%%%%%%%%%%%%%%%%%%%%%%%%%%%%%%%
\section{Filter calls}

\begin{rail}
	FilterCalls: (backslash FilterCall)*;
  FilterCall: ( FilterIdent ('(' Arguments ')'|()) | FilterIdent '<' VariableIdent '>' | 'auto' );
\end{rail}\ixnterm{FilterCalls}\ixnterm{FilterCall}\ixkeyw{auto}

Filters are employed with \verb#r\f# notation at rule calls from the sequences language.
This holds for user-implemented, auto-generated, and auto-supplied filters.

\begin{example}
The following rule \texttt{deleteNode} yields to a def variable the amount of outgoing edges of a node, and deletes the node.
Without filtering this behaviour would be pointless, but a filter ordering based on the def variable is already declared for it.
\begin{grgen}
rule deleteNode
{
	def var i:int;

	n:Node;

	yield {
		yield i = outgoing(n).size();
	}
	
	modify{
		delete(n);
	}
} \ orderAscendingBy<i>
\end{grgen}
The rule may then be applied with a sequence like the following:
\begin{grshell}
  exec [deleteNode \ orderAscendingBy<i> \ keepFirstFraction(0.5)]
\end{grshell}
This way, the 50\% of the nodes with the smallest number of outgoing edges are deleted from the host graph (because they don't get filtered away), or rephrased: the 50\% of the nodes with the highest number of outgoing edges are kept (their matches at the end of the ordered matches list are filtered away).
\end{example}
	
%\makeatletter
\begin{table}[htbp]
\centering
\begin{tabular}{|l|}
\hline
\texttt{orderAscendingBy<def var:int|double|string>}\\
\texttt{orderDescendingBy<def var:int|double|string>}\\
%\texttt{groupBy<def var:int|double|string>}\\
\texttt{keepSameAsFirst<def var:int|double|string>}\\
\texttt{keepSameAsLast<def var>:int|double|string}\\
\texttt{keepOneForEach<def var:int|double|string>}\\
\texttt{auto}\\
\hline
\texttt{keepFirst(int-Number)}\\
\texttt{keepLast(int-Number)}\\
\texttt{keepFirstFraction(double-Number)}\\
\texttt{keepLastFraction(double-Number)}\\
\hline
\end{tabular}
\caption{Auto-generated and then auto-supplied filters at a glance}
\label{filterstab}
\end{table}


