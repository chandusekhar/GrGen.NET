\chapter{Set Returning Graph Queries}\indexmain{set returning graph queries}
\label{cha:setbasedgraph}

In this chapter we'll have a look at primarily set returning graph queries, which are available as functions in the expressions of the rules, as well as in the expressions of the sequence computations.
They are primarily used to query for connected elements.
While they are convenient to use, they require to build a set that is likely thrown away thereafter, so esp. for large sets they are inefficent.

%%%%%%%%%%%%%%%%%%%%%%%%%%%%%%%%%%%%%%%%%%%%%%%%%%%%%%%%%%%%%%%%%%%%%%%%%%%%%%%%%%%%%%%%%%%%%%%%
\section{Set Based Graph Queries}\label{neighbouringelementsfunctions}

There are functions to ask for all nodes or edges of a type available: 
\begin{description}
\item[\texttt{nodes()}] returns all nodes in the graph, as set.
\item[\texttt{nodes(.)}] returns all nodes in the graph compatible to the given type, as set.
\item[\texttt{edges()}] returns all edges in the graph, as set.
\item[\texttt{edges(.)}] returns all edges in the graph compatible to the given type, as set.
\end{description}

Other functions allow to ask for the source or target node of an edge: 

\begin{description}
\item[\texttt{source(.)}] returns the source node of the given edge.
\item[\texttt{target(.)}] returns the target node of the given edge.
\end{description}

Furthermore, there are functions to query the directly neighbouring elements.
They allow to compute the set of edges incident to a node:

\begin{description}
\item[\texttt{incident(.)}] returns the set of the edges that are incident to the node given as argument value.
\item[\texttt{incident(.,.)}] as above, but only edges of the type given as second argument are contained.
\item[\texttt{indicent(.,.,.)}] as above, but only edges incident to an opposite node of the type given as third argument are contained.
\item[\texttt{incoming}] same as any of the incidents above, but restricted to incoming edges.
\item[\texttt{outgoing}] same as any of the incidents above, but restricted to outgoing edges.
\end{description}

In addition, they allow to compute the set of nodes adjacent to a node:

\begin{description}
\item[\texttt{adjacent(.)}] returns the set of the nodes that are adjacent to the node given as argument value.
\item[\texttt{adjacent(.,.)}] as above, but only nodes incident to an edge of the type given as second argument are contained.
\item[\texttt{adjacent(.,.,.)}] as above, but only nodes of the node type given as third argument are contained.
\item[\texttt{adjacentIncoming}] same as any of the adjacents above, but restricted to nodes reachable via incoming edges.
\item[\texttt{adjacentOutgoing}] same as any of the adjacents above, but restricted to nodes reachable via outgoing edges.
\end{description}

Beyond direct neighbourhood, transitive neighbourhood can be queried with the reachability functions.
They allow to compute the set of edges reachable from a node:

\begin{description}
\item[\texttt{reachableEdges(.)}] returns the set of the edges that are reachable from the node given as argument value.
\item[\texttt{reachableEdges(.,.)}] as above, but only edges of the type given as second argument are contained and followed.
\item[\texttt{reachableEdges(.,.,.)}] as above, but only edges incident to an opposite node of the type given as third argument are contained and followed.
\item[\texttt{reachableEdgesIncoming}] same as any of the reachableEdges above, but restricted to incoming edges.
\item[\texttt{reachableEdgesOutgoing}] same as any of the reachableEdges above, but restricted to outgoing edges.
\end{description}

In addition, they allow to compute the set of nodes reachable from a node:

\begin{description}
\item[\texttt{reachable(.)}] returns the set of the nodes that are reachable from the node given as argument value.
\item[\texttt{reachable(.,.)}] as above, but only nodes incident to an edge of the type given as second argument are contained and followed.
\item[\texttt{reachable(.,.,.)}] as above, but only nodes of the node type given as third argument are contained and followed.
\item[\texttt{reachableIncoming}] same as any of the reachables above, but restricted to nodes reachable via incoming edges.
\item[\texttt{reachableOutgoing}] same as any of the reachables above, but restricted to nodes reachable via outgoing edges.
\end{description}

% todo: beispiele
 