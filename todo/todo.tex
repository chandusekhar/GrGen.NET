\NeedsTeXFormat{LaTeX2e}
\documentclass[12pt,a4paper]{article}
\usepackage{german,a4,latexsym,graphicx,amssymb,color,amsmath}




\begin{document}



\newcommand{\mysection}[1]{\vspace{0.8cm}\noindent{\bf #1}\vspace{0.4cm} \\}





\noindent
{\bf\Large The GrGen.NET ToDo-List}



\mysection{Frontend (Java):}
ToDo:
\begin{itemize}
  \item Annotions of anonymous nodes and edges do not work.
  \item In replace/modify-part {\tt typeof} does not work.
  \item Dangling edge graphlets should work, if the edge is a reused one and if all incident pattern nodes of that edge are also reused.
    However, if an edge is retyped, GrGen reports an error, which should not be the case.
	\item For .grg-files, that are empty except for the {\tt actions} statement, GrGen reports an error, which should not be the case.
	\item In declarations of enumeration types (keyword {\tt enum}) user defined integer values can be assigned to elements.
	  However, on the RHS of such "`assignments"' it should be possible to use already defined elements of that enumeration types in expression (e.g., \dots{}{\tt{}x = 42, y = x + 3}\dots). However, GrGen does not accept this.
\end{itemize}
Done:
\begin{itemize}
  \item If the filename of a .grg-file does not conform with the name given along with the {\tt actions} keyword, an error is raised (which is just the right behaviour). However, the output file is genrated all the same, which should not happen.\\
    \{fixed---Batz 7/13/2007\}
  \item Error detection for the {\tt return} statement is errornous.\\
  \{seems to work now---Batz 7/13/2007\}
  \item At test case should\_fail/ret\_001.grg:
  	The signature of the rule demands a type {\tt AB}.
	However, if you return a type {\tt C} (that is no subtype of {\tt AB}) no error is reported, which were the right behaviour.\\
  \{as the return stuff now works, this works, too---Batz 7/13/2007\}
\end{itemize}



\mysection{C\#-Searchplan-Backend (Java):}
ToDo:
Done:



\mysection{Backend (C\#):}
ToDo:
Done:



\mysection{Other things like, e.g., bugs of unknown origin}
ToDo:
Done:


\end{document}
